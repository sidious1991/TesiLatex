\chapter{Test dell’\textit{edge-recommendation system} in modalità \textit{greedy e non}}
\label{chap:test}

Il sistema implementato è stato sottoposto a vari test con lo scopo di valutare i due algoritmi di \textit{k-edge recommendation} alternativi, ovvero \textit{greedy e non greedy}, ed in modo tale da poterli confrontare tra loro in termini di:
\begin{itemize} 
\item decremento totale dell'\textit{RWC} che ciascuno di essi riesce ad apportare ad un certo \textit{retweet graph} in input, a parità di numero di archi da proporre \textit{k};
\item qualità dei \textit{k} archi scelti, in termini del \textit{$\delta RWC$} associato a ciascuno di essi;
\item tempi di esecuzione.
\end{itemize}

I \textit{retweet graphs} utilizzati come input dei test corrispondono alle discussioni attorno agli \textit{hashtags} controversi \textit{\#beefban, \#indiana, \#russia\_march}, le cui informazioni (i.e. \textit{tweets} e \textit{retweets} emessi nel periodo di osservazione) sono reperibili presso il \textit{repository} degli autori dell'articolo \cite{garimella:paper}. Pertanto, in questo caso, non è stato necessario eseguire il \textit{processo di raccolta dati} descritto nel capitolo precedente ma, per ciascuno degli \textit{hashtags} appena menzionati, è bastato \textit{parsare} il file dei \textit{retweets} (disponibile nel \textit{repository}) e creare il \textit{retweet graph} corrispondente.
\\I \textit{retweet graphs} creati, relativi agli \textit{hashtags} di cui sopra, hanno le seguenti caratteristiche:
\\\\
\begin{tabular}{l*{6}{c}r}
\textbf{Hashtag}         & \textbf{|V|} & \textbf{|E|}  \\
\hline
\#beefban 		 & 1610 & 1978  \\
\#indiana        	 & 2467 & 3143  \\
\#russia\_march   	 & 2134 & 2951  \\
\end{tabular}
\\\\\\
I parametri del sistema sono stati impostati con i seguenti valori:
\begin{itemize}
\item $\alpha = 0.85$;
\item $k_1 = 20$; 
\item $k_2 = 20$;
\item $k = 50$.
\end{itemize}
Nel prossimo paragrafo, per ciascuno dei tre \textit{retweet graphs} in input, verranno mostrati e commentati i risultati dei test in termini della discesa dell'\textit{RWC} derivante dall'applicazione di ciascuno dei due algoritmi di \textit{recommendation} proposti. 

\section{Discesa dell'\textit{RWC}}

Di seguito inseriamo i grafici che mostrano la discesa dell'\textit{RWC} dei \textit{retweet graphs} relativi agli \textit{hashtags} considerati, nell'ordine:
\begin{enumerate}
\item \textit{\#beefban};
\item \textit{\#indiana};
\item \textit{\#russia\_march}.
\end{enumerate}

\includepdf{images/BEEFBAN_IN_DEG_PROBABILITY_FREE_RWC_DESCENT.pdf}
\includepdf{images/INDIANA_IN_DEG_PROBABILITY_FREE_RWC_DESCENT.pdf}
\includepdf{images/RUSSIA_MARCH_IN_DEG_PROBABILITY_FREE_RWC_DESCENT.pdf}

\section{Qualità degli archi proposti}
