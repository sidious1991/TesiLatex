\chapter{Conclusioni e sviluppi futuri}
\label{chap:conclusioni}
Entrambi gli algoritmi di \textit{k-edge recommendation} implementati hanno l'obiettivo di ridurre la controversia di una discussione che ha luogo nel \textit{social network} di \textit{Twitter} connettendo tra loro utenti che hanno opinioni e punti di vista opposti a riguardo. Tuttavia, la modalità di scelta degli utenti da coinvolgere nel processo di \textit{bridging} delle \textit{echo-chambers} che rispettivamente adottano li rende tra loro alternativi. 
\\Come è possibile dedurre dai risultati dei test, le loro differenze riguardano:
\begin{itemize}
\item i tempi di esecuzione, a parità di archi da proporre;
\item l'efficacia, in termini del decremento del grado di controversia, che a parità di archi da proporre riescono a garantire. 
\end{itemize}
Pertanto, la scelta dell'algoritmo da utilizzare deve essere dettata dalla particolare esigenza che si vuole soddisfare. Se i requisiti sui tempi di esecuzione sono molto stringenti è opportuno optare per la versione \textit{non-greedy}, a maggior ragione se il \textit{retweet graph} in input è costituito da un numero molto elevato di nodi e di archi; al contrario, se si intende privilegiare il fattore \textit{efficacia} degli archi proposti allora è opportuno scegliere la versione \textit{greedy}. Per questo il \textit{framework} implementato permette all'utente di scegliere liberamente l'algoritmo da utilizzare in fase di \textit{recommendation}.
\\Bisogna sottolineare che entrambi gli algoritmi implementati non considerano il fattore \textit{probabilità di accettazione} durante la scelta dei \textit{k} archi da proporre: tale probabilità, se fosse introdotta, terrebbe conto del fatto che non sempre gli utenti approvano mediante \textit{retweet} contenuti che esprimono opinioni opposte alle proprie e, pertanto, ognuno dei \textit{k} archi scelti non si materializza nel \textit{retweet graph} con probabilità \textit{1}. 
\\In particolare, sarebbe possibile associare ad ogni arco la propria \textit{probabilità di accettazione}, calcolata mediante un opportuno \textit{link predictor}, e modificare leggermente gli algoritmi di \textit{recommendation} implementati in modo tale che non scelgano i \textit{k} archi da proporre solamente in funzione del decremento dell'\textit{RWC} che ciascuno di essi apporterebbe al \textit{retweet graph} se si materializzasse ma anche in funzione di tale probabilità. 
\\In sintesi, in questo modo, ad ogni arco considerato bisognerebbe associare il proprio \textit{$\delta RWC$ atteso}, definito come:
\\\\
$E(\delta RWC_e) = p_e \times \delta RWC_e$
\\\\
Dove \textit{e} è l'arco diretto considerato e \textit{$p_e$} è la probabilità che esso si materializzi nel grafo (i.e. la \textit{probabilità di accettazione}). 
\\L'inclusione di tale probabilità nei due algoritmi di \textit{recommendation} proposti rappresenta senz'altro un importante sviluppo futuro del \textit{framework}: sarebbe possibile rilevare e scartare tutti quegli archi che sono promettenti dal punto di vista del proprio \textit{$\delta RWC$} ma sono caratterizzati da una \textit{probabilità di accettazione} molto bassa, cosa che gli algoritmi attualmente implementati non riescono a fare. 
Sarebbe inoltre interessante condurre nuovamente tutti i test illustrati nel capitolo precedente con l'obiettivo di confermare o confutare la tesi per cui l'algoritmo \textit{greedy} propone archi più efficaci rispetto a quelli proposti dall'algoritmo \textit{non-greedy}.
\\Per terminare, la scelta di considerare solo archi diretti che connettono i \textit{$k_1$} vertici con \textit{in degree} più alto di una \textit{echo-chamber} con i \textit{$k_2$} vertici con \textit{in degree} più alto dell'altra (e viceversa) non è da ritenersi la migliore in assoluto ma, al contrario, lascia spazio ad altre possibili euristiche, a patto che queste permettano di individuare archi più efficaci in termini del decremento del grado di controversia della discussione in atto che sono in grado di apportare. Risulta evidente che la ricerca di tali euristiche costituisce un altro ambito di sviluppo futuro, oltre all'ambito della \textit{probabilità di accettazione} già illustrato.
