%CLASSE DOCUMENTO - LINGUA E DIMENSIONE FONT
\documentclass[trieste,12pt]{toptesi}
%%%%%%%%%%%%%%%%%%%%%%%%%%%%%%%%%%%%%%%%%%%%%%%%%%%%%%%%%%%%%%%

% INCLUSIONE PACCHETTI
%\usepackage[utf8x]{inputenc}
\usepackage[utf8]{inputenc} %utf8
\usepackage[italian]{babel}
\usepackage[T1]{fontenc}
\usepackage{blindtext}
\usepackage{graphicx,wrapfig}
\usepackage{booktabs}
\usepackage{lmodern}
\usepackage{varioref}
\usepackage{url}
\usepackage{array}
\usepackage{paralist}{\obeyspaces\global\let =\space}
\usepackage{verbatim} 
\usepackage{subfig}
\usepackage{tabularx}
\usepackage{amsmath}
\usepackage{amsfonts}
\usepackage{float}
\usepackage{amssymb}
\usepackage{multicol}
\usepackage{multirow}
\usepackage{listings}
\usepackage[pass]{geometry}
\usepackage[figuresright]{rotating}
\usepackage{algorithm}
\usepackage{algorithmic}
\usepackage{amsmath}
\usepackage[babel]{csquotes}
\usepackage{hyperref}
\usepackage[backend=bibtex]{biblatex}

%%%%%%%%%%%%%%%%%%%%%%%%%%%%%%%%%%%%%%%%%%%%%%%%%%%%%%%%%%%%%%%

% CONFIGURAZIONE LINK E RIFERIMENTI
\hypersetup{%
    pdfpagemode={UseOutlines},
    bookmarksopen,
    pdfstartview={FitH},
    colorlinks,
    linkcolor={black}, %COLORE DEI RIFERIMENTI AL TESTO
    citecolor={blue}, %COLORE DEI RIFERIMENTI ALLE CITAZIONI
    urlcolor={blue} %COLORI DEGLI URL
}

%%%%%%%%%%%%%%%%%%%%%%%%%%%%%%%%%%%%%%%%%%%%%%%%%%%%%%%%%%%%%%%

% CONFIGURAZIONE LISTATI/CODICE - CANCELLARE SE NON NECESSARIO
% PYTHON - BIANCO E NERO
\lstset{%
	captionpos=b,
	language=Python,
	basicstyle =\small\ttfamily,
	keywordstyle=\color{black}\bfseries,
	breaklines=true,
	breakatwhitespace=true,
	frame=lines,
	numbers=left,
	numberstyle=\footnotesize,
}

%%%%%%%%%%%%%%%%%%%%%%%%%%%%%%%%%%%%%%%%%%%%%%%%%%%%%%%%%%%%%%%

% FRENCHSPACING VA _SEMPRE_ ABILITATO PER DOCUMENTI IN ITALIANO
\frenchspacing

%%%%%%%%%%%%%%%%%%%%%%%%%%%%%%%%%%%%%%%%%%%%%%%%%%%%%%%%%%%%%%%

%DEFINIZIONE SEZIONI IN NUMERAZIONE ROMANA
%ELENCO DEI LISTATI/CODICI
\makeatletter
\newcommand\listofcodes{%
 \iffrontmatter\else\frontmattertrue\fi
 \if@openright\cleardoublepage\else\clearpage\fi
 % change the meaning of \chapter in a group
 \begingroup\def\chapter##1{\@schapter}
 \phantomsection % for the hyperlink
 \lstlistoflistings 
 \endgroup
} 
\makeatother

%%%%%%%%%%%%%%%%%%%%%%%%%%%%%%%%%%%%%%%%%%%%%%%%%%%%%%%%%%%%%%%

% INFORMAZIONI PDF - PERSONALIZZARE
\pdfinfo{%
  /Title    (Riduzione del grado di controversia delle rete sociali connettendo punti di vista opposti)
  /Author   (Stefano Agostini)
  /Subject  (Non definito)
  /Keywords (Controversia)
}

%%%%%%%%%%%%%%%%%%%%%%%%%%%%%%%%%%%%%%%%%%%%%%%%%%%%%%%%%%%%%%%

\ateneo{Università di Roma Tor Vergata}

% LOGO UNIVERSITA
\logosede{images/Logo-Uni-Tor-Vergata.png}

\facolta{Ingegneria}

\corsodilaurea{Ingegneria Informatica}

% TIPOLOGIA TESI
\TesiDiLaurea{Tesi di Laurea Magistrale}

% TITOLO
\titolo{Studio e sviluppo di strategie per la riduzione del random-walk controversy score tra echo chambers dei social networks}

% RELATORE/I
\relatore{Giuseppe F. Italiano}
\secondorelatore{Nikos Parotsidis}

% CANDIDATO - DICITURA (MANTENERE I DUE PUNTI) - CANCELLARE O DECOMMENTARE
%\CandidateName{Candidate:}

% CANDIDATO - NOME E COGNOME
\candidato{\tabular{l}Stefano \textsc{Agostini}\\matricola: 0234240\endtabular}

% DATA - MESE ANNO
\sedutadilaurea{A.A. 2017/2018}

%%%%%%%%%%%%%%%%%%%%%%%%%%%%%%%%%%%%%%%%%%%%%%%%%%%%%%%%%%%%%%%

% LISTA DEI CAPITOLI DA INCLUDERE - PERSONALIZZARE
\includeonly{%
chap_0,%
%chap_1,%
chap_2,%
chap_3,%
chap_4,%
chap_5,%
app_a,%
app_b,%
}

% FILE DI BIBLIOGRAFIA
\bibliography{bibliography} 


%%%%%%%%%%%%%%%%%%%%%%%%%%%%%%%%%%%%%%%%%%%%%%%%%%%%%%%%%%%%%%%

% INIZIO DOCUMENTO
\begin{document}

\frontespizio

%%%%%%%%%%%%%%%%%%%%%%%%%%%%%%%%%%%%%%%%%%%%%%%%%%%%%%%%%%%%%%%

%INTERLINEA - DEFAULT 1 - NON ESAGERATE, NON SUPERATE MAI 1.3 ;)
\interlinea{1.5}

%%%%%%%%%%%%%%%%%%%%%%%%%%%%%%%%%%%%%%%%%%%%%%%%%%%%%%%%%%%%%%%

\frontmatter

% DEDICA - PERSONALIZZARE
% VSPACE - PROPORZIONE USATA PER CENTRATURA VERTICALE DEL TESTO
% FLUSHRIGHT - ALLINEAMENTO ORIZZONTALE A DESTRA
\vspace*{\stretch{1}}
\begin{flushright}
\noindent
Da inserire qui tutte le dediche della mia tesi...
\end{flushright}
\vspace*{\stretch{6}}
\cleardoublepage

%%%%%%%%%%%%%%%%%%%%%%%%%%%%%%%%%%%%%%%%%%%%%%%%%%%%%%%%%%%%%%%

% ABSTRACT - PERSONALIZZARE
\sommario
Esistono, e sono sempre esistiti, particolari temi, detti \textit{controversi}, per i quali ognuno di noi si schiera come sostenitore o come oppositore; tali temi possono riguardare contesti politici, sociali o culturali. L'effetto quasi immediato è la divisione della popolazione in due gruppi che hanno visioni opposte sull'argomento \textit{controverso} in considerazione e che difficilmente scambiano tra loro informazioni e punti di vista, non facendo altro che acutizzare la loro inconciliabilità di opinione: gruppi di individui di questo genere sono anche detti \textit{echo chambers}. Gli invidui facenti parte della stessa \textit{echo chamber} hanno quindi le stesse credenze e lo stesso parere riguardo all'argomento \textit{controverso}: essi rinforzano a vicenda le proprie opinioni e sono scarsamente esposti a punti di vista opposti ai propri (i.e. le opinioni che caratterizzano l'altra \textit{echo chamber}). Non è immune da tale fenomeno il mondo dei \textit{social media}. In particolare, \textit{Twitter} identifica un particolare \textit{topic} (i.e. argomento) mediante un \textit{hashtag} (e.g.\textit{\#novax}) e la discussione riguardo a tale \textit{topic} può essere descritta mediante un \textit{endorsement graph}, ovvero un \textit{grafo diretto} in cui ciascun nodo rappresenta un utente che partecipa alla discussione e vi è un arco diretto da un nodo \textit{x} ad un nodo \textit{y} se e solo se l'utente \textit{x} approva l'opinione contenuta in un \textit{tweet} dell'utente \textit{y} (dove l'approvazione è espressa per mezzo dello strumento del \textit{retweet}). Quando il \textit{topic} in considerazione è \textit{controverso}, la struttura dell'\textit{endorsement graph} mette in luce la presenza di gruppi (di utenti) molto connessi al loro interno ma che comunicano poco tra loro: la distanza di opinione che separa tali gruppi (o \textit{echo-chambers}) viene quantificata mediante il così detto \textit{indice di controversia} del grafo, che viene misurato, nel lavoro di tesi proposto, utilizzando una metrica basata sui \textit{random walks}. \textit{Endorsement graphs} con elevati \textit{indici di controversia} sono caratterizzati da \textit{echo chambers} poco connesse tra loro, in ciascuna delle quali è amplificata una visione univoca ed acritica sull'argomento \textit{controverso} considerato.\\Con l'obiettivo di ridurre efficacemente tale \textit{indice di controversia}, è stato implementato un \textit{framework}, utilizzando il linguaggio \textit{Python} e con l'ausilio della libreria \textit{NetworkX}, il quale:

\begin{enumerate}
\item acquisisce i dati necessari per costruire l'\textit{endorsement graph} associato ad un \textit{hashtag controverso}, presente nel \textit{social network} di \textit{Twitter}, fornito in input. Cattura la collezione dei dati necessari mediante la libreria \textit{Python Tweepy} che accede all'\textit{API} di \textit{Twitter};
\item esegue un algoritmo che identifica le \textit{echo chambers};
\item implementa un \textit{edge-recommendation system} che permette di individuare \textit{k} archi diretti (\textit{k} fornito in \textit{input}) che, se aggiunti al grafo, riducono il suo \textit{grado di controversia}. Nella pratica si vuole esporre alcuni utenti al contenuto di altri sperando che lo approvino mediante \textit{retweet}: l'effetto di questa approvazione nell'\textit{endorsement graph} sarebbe la comparsa degli archi consigliati;
\item offre un \textit{tool} di visualizzazione degli archi individuati all'interno del grafo.
\end{enumerate}
Con riferimento al punto \textit{3}, il problema di ottimizzazione che si vorrebbe risolvere sarebbe quello di trovare il \textit{set} di \textit{k} archi diretti che, se aggiunti al grafo, minimizzano il \textit{grado di controversia}.
Poiché gli \textit{endorsement graphs} dei \textit{social networks} sono generalmente costituiti da un numero molto elevato di nodi (indicato con \textit{n}), risolvere tale problema di ottimizzazione considerando tutte le possibili combinazioni degli archi a gruppi di \textit{k} (complessità \textit{O}(${n^2\choose k}$)) (approccio \textit{brute force}) è evidentemente molto costoso dal punto di vista computazionale e molto inefficiente anche per quanto riguarda i tempi di esecuzione. Pertanto l'approccio seguito è quello adottato nell'articolo \textit{"Reducing controversy by connecting opposing views"}\cite{garimella:paper}, che consiste nel considerare solo un sottoinsieme degli archi possibili (i.e. un sottoinsieme degli archi non ancora materializzati nell'\textit{endorsement graph}) ed estrarre da questo sottoinsieme i \textit{k} più promettenti. Chiaramente questa soluzione potrebbe restituire archi meno efficaci, per quanto riguarda la riduzione del \textit{grado di controversia} che consentono, rispetto a quelli restituiti dall'approccio \textit{brute force} ma apporta un miglioramento in termini di efficienza computazionale; in particolare, l'euristica che specifica la modalità di scelta del sottoinsieme degli archi candidati è cruciale. L'euristica utilizzata in questo lavoro di tesi è quella proposta nell'articolo \cite{garimella:paper}, la quale
consiste nel considerare solo gli archi diretti che permettono di connettere i vertici di grado alto della prima comunità (o \textit{echo chamber}) con i vertici di grado alto della seconda comunità e viceversa: da questo sottoinsieme di archi vengono estratti i \textit{k} più promettenti in termini di riduzione del \textit{grado di controversia} che consentono.\\La bontà di tale euristica è stata valutata considerando due algoritmi alternativi, utilizzati per estrarre i \textit{k} archi più promettenti dal sottoinsieme considerato: 
\begin{itemize}
\item \textit{non-greedy}: vengono scelti in un solo \textit{step} i \textit{k} archi che porterebbero al \textit{grado di controversia} più basso qualora venissero aggiunti al grafo \textit{individualmente};
\item \textit{greedy}: vengono scelti \textit{k} archi in \textit{k steps}, in ognuno dei quali viene estratto l'arco migliore, tra quelli rimanenti, in termini di decremento del \textit{grado di controversia} che apporterebbe se fosse aggiunto al grafo.
\end{itemize}
Nel lavoro di tesi verranno descritte le modalità di implementazione di tali algoritmi e successivamente verranno confrontati tra loro in termini di efficacia (ossia in termini del decremento del \textit{grado di controversia} che consentono, qualora tutti gli archi che consigliano venissero accettati) ed in termini di tempi di esecuzione ed efficienza computazionale: a tal fine sono stati condotti \textit{tests} su tre \textit{endorsement graphs} di \textit{Twitter} corrispondenti ad \textit{hashtags} particolarmente controversi (\textit{\#beefban,\#russia\_march,\#indiana}), volutamente scelti da contesti sociali e culturali diversi in modo tale da ottenere un'analisi più attendibile.\\L'algoritmo \textit{greedy} si rivelerà più preciso in quanto ad ogni \textit{step} si limita a proporre uno ed un solo arco, ossia l'arco migliore in termini del decremento del \textit{grado di controversia} che consentirebbe se fosse accettato; inoltre, ad ogni \textit{step} dell'algoritmo la scelta dell'arco migliore viene condotta solo dopo aver aggiunto al grafo tutti gli archi consigliati negli \textit{steps} precedenti.\\Al contrario, l'algoritmo \textit{non-greedy} propone in un solo passo i \textit{k} archi migliori utilizzando come metrica il decremento del \textit{grado di controversia} che ciascuno di essi apporterebbe se fosse aggiunto \textit{individualmente}: poichè viene valutato il loro impatto \textit{individuale} e viene ignorato il fatto che tale impatto potrebbe decrementare rispetto a quanto valutato man mano che essi vengono aggiunti al grafo, quest'algoritmo rappresenta un'approssimazione dell'algoritmo \textit{greedy} e consente, in generale, un decremento minore del \textit{grado di controversia}. D'altra parte l'algoritmo \textit{greedy} richiede di scansionare tutti gli archi considerati \textit{k} volte (una volta per \textit{step})  mentre l'algoritmo \textit{non-greedy} una volta sola: questo si traduce in un sostanziale vantaggio dal punto di vista dei tempi di esecuzione, ottenibile utlilizzando l'algoritmo \textit{non-greedy}.\\I test condotti sugli \textit{endorsement graphs} di \textit{Twitter} considerati mostreranno che, spesso, l'algoritmo \textit{non-greedy} conduce a risultati confrontabili con quelli ottenuti dall'algoritmo \textit{greedy} e, pertanto, vista la sua maggiore efficienza computazionale può risultare una scelta più vantaggiosa.
\\Per finire, va sottolineato che, in generale, gli archi che vengono scelti dall'\textit{edge-recommendation system} nella realtà non sempre si materializzano (l'utente potrebbe rigettare il consiglio) e per questo è opportuno considerare come metrica anche la \textit{probabilità di accettazione}. Con il proposito di future estensioni, compresa l'introduzione di tale \textit{probabilità}, il \textit{framework} proposto è implementato in modo da prestarsi perfettamente all'aggiunta di altre metriche per la scelta degli archi.



%%%%%%%%%%%%%%%%%%%%%%%%%%%%%%%%%%%%%%%%%%%%%%%%%%%%%%%%%%%%%%%

% INDICI - ELIMINARE GLI INDICI NON NECESSARI

% INDICE GENERALE
\tableofcontents

% INDICE DELLE FIGURE
\listoffigures

% INDICE DELLE TABELLE
\listoftables

% INDICE DEI CODICI
\listofcodes

%%%%%%%%%%%%%%%%%%%%%%%%%%%%%%%%%%%%%%%%%%%%%%%%%%%%%%%%%%%%%%%

\mainmatter

% INCLUSIONE FILE CAPITOLI - PERSONALIZZARE - TENERE COERENTE CON LISTA IN ALTO
\chapter{Introduzione}
\label{chap:introduzione}

Il fenomeno della \textit{polarizzazione} degli utenti attorno a \textit{topic controversi} che si propongono nei \textit{social media} è ben noto ed il suo studio è già stato affrontato in alcuni articoli scientifici (tra cui \cite{musco:paper}\cite{morales:paper}). Potremmo definire la \textit{polarizzazione} come segue:
\\\\
\textit{Situazione che determina la divisione della popolazione in gruppi con punti di vista opposti riguardo ad un certo argomento.}
\\\\Molto spesso i \textit{social media}, mediante algoritmi di \textit{recommendation}, espongono gli utenti solo a contenuti che si addicono e sono conformi alle loro opinioni e, pertanto, non fanno altro che aggravare il loro stato di polarizzazione. Tutto ciò determina la formazione delle così detto \textit{echo-chambers}, ossia:
\\\\
\textit{Situazioni in cui individui che hanno lo stesso parere su un certo argomento rafforzano l'opinione reciproca ma non vengono esposti ad opinioni opposte alla propria.}
\\\\
Lo scopo di questa tesi è quello di sviluppare un sistema in grado far comunicare, nel modo più efficace possibile, queste \textit{echo-chambers} così da esporre gli individui a punti di vista opposti ai propri e ridurre la \textit{controversia} della discussione. L'ambiente in cui opera il \textit{framework} proposto è il \textit{social network} di \textit{Twitter}, in cui gli argomenti delle discussioni vengono identificati da \textit{hashtags}, i contenuti relativi vengono espressi dagli utenti attraverso i \textit{tweets} e le condivisioni di opinione attraverso i \textit{retweets}. Una discussione nell'ambiente di \textit{Twitter} può essere descritta mediante un \textit{endorsement graph}, ossia un grafo i cui nodi sono utenti che hanno espresso almeno un'opinione mediante un \textit{tweet} ed i cui archi rappresentano i \textit{retweets}\footnote{Vi è un arco da un nodo \textit{u} ad un nodo \textit{v} se e solo se l'utente \textit{u} ha \textit{retweettato} almeno un \textit{tweet} di \textit{v}.}.\\Il \textit{framework} acquisisce da \textit{Twitter} i dati necessari per elaborare \textit{endorsement graphs} di \textit{topic controversi}, ne analizza la struttura estraendone le \textit{echo chambers} ed infine implementa un \textit{edge-recommendation system} che ha lo scopo di ridurre il \textit{grado di controversia} creando connessioni (\textit{bridges}) tra utenti che hanno punti di vista opposti (appartengono a \textit{echo chambers} distinte). Infatti la naturale propensione degli individui a dare credito solo a notizie e contenuti che si addicono al proprio parere fa sì che, in assenza di un intervento esterno di \textit{edge-recommendation}, essi rafforzino sempre più la propria convinzione, anche qualora questa fosse sbagliata o acritica.\\\\Il sistema proposto in questo lavoro di tesi si compone di più fasi successive, attraverso le quali raggiunge l'obiettivo preposto; nel seguito vengono descritte sinteticamente tali fasi, fornendo le nozioni teoriche e le assunzioni sulle quali si fonda la loro implementazione.
\\\\La prima fase si occupa dell'acquisizione dei dati da \textit{Twitter} e quindi della costruzione dell'\textit{endorsement graph} associato. Il \textit{framework} utilizza la libreria \textit{Python Tweepy} (che accede all'\textit{API} di \textit{Twitter}) per ottenere tutti i \textit{tweets} e \textit{retweets} emessi dagli utenti, riguardanti l'\textit{hashtag} fornito in input, in un certo intervallo di tempo. Al termine della collezione di tali dati, essi vengono \textit{parsati} per la costruzione dell'\textit{endorsement graph} che descrive la discussione. Gli \textit{endorsement graphs}, nel particolare ambiente di \textit{Twitter}, prendono anche il nome di \textit{retweet graphs}. Dato un \textit{hashtag} controverso, viene a formarsi naturalmente una discussione a riguardo, nella quale gli utenti esprimono una propria opinione e possono approvare il punto di vista delle così dette \textit{autorità}: nel caso particolare di \textit{Twitter} questa \textit{approvazione} si realizza mediante lo strumento del \textit{retweet}, ossia se l'utente \textit{u} fa \textit{retweet} di un \textit{tweet} prodotto dall'utente \textit{v} allora ne approva l'opinione. Ne deriva la formazione di un grafo diretto \textit{G(V,E)} costituito da \textit{n} nodi (gli utenti che partecipano alla discussione) ed i cui archi (\textit{retweets}) esprimono relazioni di condivisione di punti di vista: proprio questo grafo rappresenta l'\textit{output} di questa fase.
\\\\La seconda fase si occupa di rilevare le \textit{echo-chambers} dell'\textit{endorsement graph} in input e di calcolarne il \textit{grado di controversia}. Le \textit{echo-chambers} sono due sottoinsiemi dei nodi del grafo \textit{X,Y}, ben separati tra loro (vi sono pochi archi che li congiungono) e tali che \textit{$X \cup Y = V$} e \textit{$X \cap Y =  \emptyset$}. Tale ripartizione dei nodi può essere ottenuta mediante l'utilizzo di un algoritmo di \textit{graph-partitioning}. Nel lavoro proposto è stato utilizzato l'algoritmo di \textit{Girvan-Newman}: esso è un metodo gerarchico usato per rilevare le comunità in sistemi complessi e la cui esecuzione produce un \textit{dendrogramma}, le cui foglie sono i nodi del grafo. Individuate le \textit{echo-chambers}, questa fase si occupa di quantificare la \textit{controversia} della discussione.
\\Il sistema sviluppato utilizza una metrica basata sul concetto di \textit{random walk} per misurare il \textit{grado di controversia} associato al \textit{topic} analizzato (attorno al quale si svolge la discussione nel \textit{social network} di \textit{Twitter}). Più precisamente, per misurare il \textit{grado di controversia} della rete, viene utilizzata la funzione \textit{Random-walk controversy score}:
\\\\
$\textit{RWC(G,X,Y) = (c\textsubscript{x} - c\textsubscript{y})\textsuperscript{T}(r\textsubscript{x} - r\textsubscript{y})}$\footnote{La definizione del \textit{Random-walk controversy score} è tratta dall'articolo \cite{garimella:paper}.}
\\\\Dove \textit{c\textsubscript{x}} è un vettore di dimensione \textit{n} (numero di vertici dell'\textit{endorsement graph}) che ha valore \textit{1} nelle coordinate corrispondenti ai vertici di grado alto dell'\textit{echo-chamber X} e \textit{0} altrove; similmente viene definito \textit{c\textsubscript{y}}. Infine \textit{r\textsubscript{x}} è il vettore di \textit{PageRank} personalizzato per un \textit{random walk} che parte dai nodi dell'\textit{echo-chamber X}; similmente viene definito \textit{r\textsubscript{y}}. Valori alti di \textit{RWC(G,X,Y)} indicano che, all'equilibrio del \textit{random walk}, è bassa la probabilità di essere nell'\textit{echo-chamber} opposta a quella di partenza: questo è indice di elevata \textit{controversia}. 
\\\\La terza fase ha lo scopo di ridurre il \textit{grado di controversia} rilevato nella fase precedente mediante l'esecuzione di un \textit{edge-recommendation system}. In particolare, il sistema implementato permette, utilizzando in alternativa un algoritmo \textit{greedy} o uno \textit{non-greedy}, di proporre \textit{k} archi (che farebbero da \textit{bridges} tra le due \textit{echo chambers}) in grado di ridurre questo indice; l'\textit{edge-recommendation system} proposto restituisce archi la cui efficacia approssima quella degli archi che sono soluzione del problema di ottimizzazione originario che, come detto precedentemente, ha una complessità di un livello troppo elevato (\textit{O}(${n^2\choose k}$)) per essere risolto in tempi accettabili. Di seguito la definizione del problema di ottimizzazione originario:
\\
\begin{equation*}
\begin{aligned}
& \underset{E\textsubscript{k}}{\text{minimize}}
& & RWC(G(V, E \cup E\textsubscript{k}),X,Y) \\
& \text{subject to}
& & E\textsubscript{k} \subseteq V \times V \setminus E, \left|{E\textsubscript{k}}\right| = k
\end{aligned}
\end{equation*}
\\
Ossia il problema originario consiste nel trovare l'insieme di \textit{k} archi, considerando \textit{tutti} gli archi non ancora presenti nel grafo, che se si materializzassero \textit{minimizzerebbero} l'\textit{indice di controversia}.
L'euristica proposta in questo lavoro permette di restringere il dominio degli archi considerati, consentendo di ottenere risultati paragonabili con quelli ottenibili mediante la soluzione del problema di ottimizzazione appena descritto e con il vantaggio di avere una complessità computazionale di molto inferiore; essa, forniti in \textit{input} i valori di \textit{k\textsubscript{1}} e \textit{k\textsubscript{2}} (interi positivi):
\begin{enumerate}
\item considera i \textit{k\textsubscript{1}} vertici con \textit{in-degree}\footnote{Il numero di archi diretti del grafo che hanno come nodo di destinazione un nodo \textit{x} è detto \textit{in-degree} di \textit{x}.} più alto dell'\textit{echo-chamber X} e i \textit{k\textsubscript{2}} vertici con \textit{in-degree} più alto dell'\textit{echo-chamber Y};
\item costruisce il dominio degli archi considerati come l'insieme di tutti i possibili archi diretti, non presenti ancora nel grafo, che abbiano come estremi un vertice dell'insieme \textit{K\textsubscript{1}} e uno dell'insieme \textit{K\textsubscript{2}};  
\end{enumerate}
Gli algoritmi \textit{greedy} e \textit{non-greedy} operano entrambi a partire dal dominio così definito. Hanno lo scopo di proporre i \textit{k} archi più promettenti del dominio in termini del decremento del grado di controversia \textit{RWC(G,X,Y)} che consentirebbero qualora si materializzassero nel grafo; la differenza sta nella modalità di selezione. In particolare:
\begin{enumerate}
\item l'algoritmo \textit{greedy} sceglie \textit{avidamente} ognuno dei \textit{k} archi: esso impiega \textit{k} passi, in ognuno dei quali sceglie l'arco più appetibile del dominio (ossia l'arco del dominio non ancora presente nel grafo associato al $\delta RWC$ maggiore) e lo aggiunge al grafo.
\item l'algoritmo \textit{non-greedy} ordina una sola volta tutti gli archi del dominio secondo il $\delta RWC$ che ognuno consente e sceglie in un solo passo i \textit{k} archi migliori. Ne deriva una maggiore efficienza nei tempi di esecuzione ma una minore precisione nella scelta degli archi da proporre.
\end{enumerate}
%qui descrivo sinteticamente i due algoritmi
In pratica, per ridurre la \textit{controversia}, si propone ad un certo insieme di utenti il contenuto (i.e. \textit{tweets}) di utenti che hanno posizioni sull'argomento opposte rispetto alle proprie, sperando che la maggior parte di loro possa farne \textit{endorsement} mediante lo strumento del \textit{retweet}: ciò provocherebbe la formazione di nuovi archi tra le due comunità (\textit{echo chambers}) con l'effetto di ridurre il \textit{grado di controversia} dell'intero \textit{retweet graph}.  
\\\\Il \textit{framework} implementato è stato testato su tre \textit{retweet graphs} di tre \textit{topics} (ossia \textit{hashtags}) controversi di \textit{Twitter}: \textit{\#beefban,\#russia\_march,\#indiana}. I test sono stati prodotti seguendo l'approccio del \textit{paper}\cite{garimella:paper}, che ha lo scopo di mostrare il livello di \textit{controversia} del grafo in funzione della quantità di archi al momento aggiunti. Tuttavia l'obiettivo dei test, in questa tesi, è in primo luogo quello di fornire un confronto dell'efficacia dell'\textit{edge-recommendation system} basato sull'algoritmo \textit{greedy} con l'efficacia dell'\textit{edge-recommendation system} basato sull'algoritmo \textit{non-greedy}. Per \textit{efficacia} si intende:
\\\\
\textit{Fissato come obiettivo il decremento del grado di controversia di una quantità $\Delta RWC$, la quantità minima di archi, consigliati dal sistema, che gli utenti devono accettare per poter raggiungerlo.}
\\\\
In secondo luogo i test hanno lo scopo di fornire un raffronto dei tempi di esecuzione dei due algoritmi al variare dell'entità del grafo in input, entità espressa in termini del numero di nodi \textit{n} e del numero di archi \textit{e}.
\\\\Per terminare, il \textit{framework} offre un \textit{tool} per la visualizzazione degli archi proposti, evidenziando le caratteristiche dei nodi coinvolti tra cui l'\textit{in degree} (ossia il grado in ingresso); inoltre è utilizzato un algoritmo di \textit{coloring}\footnote{Con \textit{coloring} si intende una colorazione esatta dei vertici, cioè un'etichettatura dei vertici del grafo con colori tali che nessuna coppia di vertici che condividono lo stesso arco abbiano lo stesso colore.} per classificare tali nodi colorandoli in modo diverso, in funzione dell'\textit{echo-chamber} a cui appartengono.
\\Di seguito è illustrata brevemente l'organizzazione della tesi.
\section{Organizzazione della tesi}
\label{sec:organizzazione}
L'esposizione del lavoro di tesi ha l'obiettivo di fornire \textit{dettagli} riguardanti:
\begin{itemize}
%\item \textit{lo stato dell'arte;}
\item \textit{la teoria che è alla base del problema affrontato;}
\item \textit{la raccolta dei dati e l'implementazione del framework;}
\item \textit{le modalità in cui sono stati effettuati i test ed i risultati ottenuti;}
\item \textit{sviluppi futuri.}
\end{itemize}
%Lo stato dell'arte verrà trattato nell'omonimo capitolo. 
Gli approfondimenti teorici verranno illustrati nel capitolo \textit{Teoria alla base del problema ed algoritmi per la risoluzione}. Il capitolo \textit{Raccolta dati ed implementazione} si occuperà di dare dettagli sulle tecnologie utilizzate per la raccolta dei dati e di illustrare i dettagli implementativi. I risultati sperimentali ottenuti dai test condotti (nelle modalità \textit{greedy e non}) e le osservazioni ad essi riguardanti sono trattati nel capitolo \textit{Test dell'edge-recommendation system in modalità greedy e non}. Il capitolo \textit{Conclusioni e sviluppi futuri} tratterà le conclusioni tratte dai risultati dei test ed approfondirà le sfide ed i propositi di miglioramento del sistema implementato.

%This is a reference to a chapter \ref{chap:quo}. This is a reference to a figure \ref{fig:doge}. This is a reference to some code \ref{lst:hello}. This is a citation %\cite{famous:paper}.

%\lstinputlisting[basicstyle=\fontsize{8}{10}\selectfont\ttfamily ,label=lst:prova, firstline=1, lastline=34, caption={I directly included a portion of a file}]{code/prova.py}

%\begin{lstlisting}[language=Java, label=lst:java, caption={Some code in another language than the default one}]
%public void prepare(AClass foo) {
%        AnotherClass bar = new AnotherClass(foo)
%}
%\end{lstlisting}

%\Blindtext

%\begin{figure}
%\begin{center}
%\includegraphics[width=0.5\columnwidth]{images/doge.png}
%\end{center}
%\caption{This is not a figure. It's a caption.}
%\label{fig:doge}
%\end{figure}

%\chapter{Lo stato dell'arte}
\label{chap:stato}
Nonostante il \textit{Web} offra metodi immediati per accedere a qualsiasi tipo di informazione, molti studi (e.g. \textit{Liao et al.}\cite{liao:paper}) dimostrano che, quando un utente può scegliere, esso preferisce essere esposto a contenuti che non contrastano con la sua opinione. Questo fenomeno ha portato ad una maggiore frammentazione e polarizzazione online.

\chapter{Teoria alla base del problema ed algoritmi per la risoluzione}
\label{chap:teoria}
\section{Misura del grado di controversia}
Prima di dare una definizione formale del \textit{random-walk controversy score}, elenchiamo ed illustriamo i passi necessari per calcolarlo.
\begin{enumerate}
\item Fissato il \textit{topic t} per il quale si vuole quantificare il grado di controversia, è possibile descrivere la discussione mediante l'\textit{endorsement graph G(V,E)}. Nell'ambiente di \textit{Twitter}, il \textit{topic t} è identificato da un \textit{hashtag} (e.g.\textit{\#hashtag}) ed i nodi del grafo rappresentano gli utenti che hanno preso parte alla discussione utilizzando almeno una volta tale \textit{hashtag} nei loro \textit{tweets}; gli archi del grafo identificano i \textit{retweets} tra gli utenti, che esprimono relazioni di condivisione di opinione riguardo il \textit{topic}.
\item Ipotizzando che il \textit{topic t} sia controverso, è possibile partizionare i nodi del grafo \textit{G(V,E)} in due insiemi \textit{X},\textit{Y} ben separati tra loro (i.e. vi sono pochi archi che li interconnettono). Tali insiemi quindi soddisfano le seguenti proprietà:
\begin{enumerate}
\item $\textit{X}\cup\textit{Y} = \textit{V}$;
\item $\textit{X}\cap\textit{Y} = \emptyset$.
\end{enumerate}
Gli insiemi \textit{X} ed \textit{Y} rappresentano i due lati della controversia (i.e. le \textit{echo-chambers}).
\\Per identificare le \textit{echo-chambers}, nell'implementazione proposta è stato utilizzato l'algoritmo di \textit{graph-partitioning} di \textit{Girvan-Newman}. Tale algoritmo agisce rimuovendo progressivamente archi dal grafo originario: l'esecuzione viene arrestata quando la rimozione degli archi ha portato ad individuare due comunità distinte che non comunicano (i.e. non sono collegate da nessun arco). La metrica utilizzata da \textit{Girvan-Newman} per identificare l'arco da rimuovere ad ogni passo è la così detta \textit{edge-betweenness centrality}: dato un arco \textit{e}, essa è definita come \textit{il numero di cammini di costo minimo tra coppie di nodi del grafo che passano attraverso l'arco e}. Nel caso in cui vi sia più di un percorso di costo minimo tra una coppia di nodi, a ciascun percorso viene assegnato uguale peso in modo tale che il peso totale di tutti i percorsi sia uguale all'unità. 
\end{enumerate}

\section{Definizione formale degli algoritmi per la risoluzione}
\section{Calcolo del decremento della controversia associato ad un arco}

\chapter{Raccolta dati ed implementazione}
\label{chap:implementazione}

Questo capitolo si occuperà, prima di tutto, di fornire dettagli sulle tecnologie e sulle modalità di raccolta dei dati da \textit{Twitter}, necessari alla costruzione di \textit{retweet graphs} associati ad \textit{hashtags} in \textit{input}.
\\Infine verrà trattata puntualmente l'implementazione degli algoritmi, definiti rigorosamente nel capitolo precedente, e di tutte quelle tecniche che hanno permesso di raggiungere l'obiettivo preposto, ovvero l'implementazione di un \textit{framework} che permetta di:
\begin{enumerate}
\item costruire ed analizzare \textit{retweet graphs} associati ad \textit{hashtags di Twitter} in \textit{input};
\item rilevare le \textit{echo-chambers} che caratterizzano la discussione;
\item eseguire un algoritmo di \textit{k-edge recommendation}, in modalità \textit{greedy} o meno;
\item fornire strumenti per l'analisi degli archi consigliati e per la visualizzazione dei nodi coinvolti (i.e. i nodi estremi degli archi consigliati).  
\end{enumerate}

\section{Raccolta dati}
La raccolta dei dati è una fase indispensabile, una condizione \textit{sine qua non}, senza la quale non è pensabile raggiungere alcun obiettivo tra quelli prefissati.
\\Poichè il \textit{software} proposto si occupa di \textit{endorsement graphs} della \textit{social network} di \textit{Twitter}, per la raccolta dei dati si è reso necessario l'utilizzo della \textit{Twitter Api}. Inoltre, visto che il linguaggio utilizzato per l'implementazione è \textit{Python}, sono state sfruttate le funzionalità della libreria \textit{Tweepy}, una via di accesso alla \textit{Twitter Api} di facile utilizzo. 
\\Nel seguito saranno forniti dettagli sugli strumenti di \textit{Twitter Api e Tweepy} e su altre tecniche che hanno permesso di \textit{bypassare} importanti limitazioni temporali delle \textit{Api di Twitter}.  

\subsection{Twitter Api}
\textit{Twitter} mette a disposizione degli sviluppatori delle \textit{Api}, utili per l'acquisizione dei dati pubblicati dagli utenti. Per rendere possibile il loro utilizzo bisogna innanzitutto creare un \textit{account Twitter} e poi effettuare l'iscrizione al \textit{reparto sviluppatori di Twitter}. Questa procedura è molto rigida e qualora non fosse seguita in modo rigoroso non sarebbe possibile utilizzare le \textit{Api}. 
\\Una volta effettuata l'iscrizione al \textit{reparto sviluppatori di Twitter}, è finalmente possibile procedere con la raccolta dei dati pubblicati dagli utenti, ma non prima di aver ottenuto le credenziali di accesso. Le credenziali vengono rilasciate a seguito della creazione di una \textit{Twitter App}, che rappresenta un \textit{progetto Twitter} dello sviluppatore: esse permetteranno di autenticarsi presso un \textit{server di Twitter}, con il quale sarà possibile interagire \textit{via streaming} mediante le \textit{Twitter Api} per ottenere i dati richiesti. Le credenziali si dividono in \textit{Token} e \textit{Consumer}, i quali hanno le seguenti caratteristiche e funzioni:
\begin{itemize}
\item il \textit{Token} permette l'accesso ai servizi che offre \textit{Twitter}. Non è sufficiente per consentire lo \textit{streaming} dei dati dal \textit{server}. \`E costituito da:
\begin{itemize}
\item \textit{Access Token};
\item \textit{Access Secret}.
\end{itemize}
\item il \textit{Consumer} consente lo \textit{streaming} dei dati di \textit{Twitter} dal \textit{server}. \`E costituito da:
\begin{itemize}
\item \textit{Consumer Key};
\item \textit{Consumer Secret}.
\end{itemize}
\end{itemize}
Nell'ambito della stessa \textit{Twitter App}, queste chiavi possono essere rigenerate a piacimento, anche con l'obiettivo di evitare problematiche relavite alla sicurezza. Qualunque sia il linguaggio di programmazione utilizzato per lo sviluppo del \textit{software} (nel caso in esame \textit{Python}), per utilizzare le \textit{Api} da codice è sempre necessario prima autenticarsi, fornendo tutti e quattro i codici appena elencati: le interfacce d'accesso alle \textit{Api} di \textit{Twitter} tuttavia dipendono dal linguaggio di programmazione e, nel nostro caso, sono realizzate mediante la libreria di \textit{Python Tweepy}, della quale parleremo nel seguito della trattazione.
\\Ad ogni modo, bisogna sottolineare che lo \textit{streaming} dei dati viene limitato da \textit{Twitter} per evitare che gli sviluppatori utilizzino in modo sconveniente i dati pubblicati dagli utenti: uno sviluppatore, pur essendo dotato di tutte le credenziali necessarie, non può effettuare più di 100 richieste ogni 15 minuti. Durante il tempo di pausa, che viene fatto scattare in corrispondenza del superamento della soglia di richieste, lo sviluppatore può decidere di aspettare che esso si esaurisca o, al contrario, può decidere deliberatamente di violarlo ed effettuare una nuova richiesta: in tal caso le sue credenziale verrebbero bloccate e non gli sarebbe permesso di comunicare con il \textit{server} mediante le \textit{Api} per circa un'ora.
\\Un'altra limitazione che impone l'\textit{Api} ufficiale di \textit{Twitter} riguarda l'impossibilità di ottenere \textit{tweets} più vecchi di una settimana: questa limitazione è molto forte ed ha costituito, nel processo di sviluppo del \textit{framework} proposto, un problema molto ingente, la cui risoluzione è dovuta alla libreria \textit{GetOldTweets di Python}, della quale parleremo presto.
\\Per terminare, i dati che lo sviluppatore richiede al \textit{server} mediante la \textit{Twitter Api} vengono restituiti in un file \textit{JSON}: esso conterrà tutti i \textit{metadati} necessari, i quali dipendono dal criterio della \textit{query}, come ad esempio il testo del \textit{tweet}, gli \textit{hashtags}, lo \textit{username} dell'utente che l'ha pubblicato e gli utenti che l'hanno \textit{retweettato}.

\subsection{Tweepy}
\textit{Tweepy} è una libreria di \textit{Python} che permette di accedere agevolmente alle \textit{Api} di \textit{Twitter}. Gestisce l'autenticazione dello sviluppatore presso il server di \textit{streaming} utilizzando i seguenti metodi:
\begin{enumerate}
\item \textit{tweepy.OAuthHandler(CONSUMER\_KEY, CONSUMER\_SECRET)}: 
\\una volta forniti \textit{Consumer Key e Consumer Secret} validi, restituisce un codice di autenticazione \textit{auth}; 
\item \textit{auth.set\_access\_token(ACCESS\_TOKEN, ACCESS\_TOKEN\_SECRET)}: 
\\permette di impostare il codice \textit{auth} con gli \textit{Access Token e Access Token Secret} (validi);
\item \textit{tweepy.API(auth)}: restituisce, in caso di corretta autenticazione, un oggetto \textit{Api} attraverso il quale può finalmente avvenire il processo di \textit{streaming} dal \textit{server}.
\end{enumerate}
Inoltre \textit{Tweepy} permette di gestire vari tipi di errore tra cui \textit{RateLimitError}, che insorge quando viene superata la soglia di traffico di 100 richieste ogni 15 minuti.

\subsection{GetOldTweets}
Come precedentemente detto, l'\textit{Api} ufficiale di \textit{Twitter} rende impossibile, con un semplice \textit{account} gratuito, l'acquisizione di \textit{tweets} più vecchi di una settimana. Questa limitazione, nel caso del sistema proposto, è intollerabile, visto che, per costruire \textit{retweet graphs} di dimensioni sufficienti a condurre un'analisi significativa, bisogna utilizzare un intervallo di osservazione abbastanza ampio. Superare tale limitazione continuando ad utilizzare l'\textit{Api} ufficiale vorrebbe dire pagare per ottenere un \textit{account Enterprise}, cosa che non siamo disposti a fare. 
\\La libreria \textit{GetOldTweets} permette di \textit{bypassare} l'\textit{Api} ufficiale e di ottenere \textit{tweets} più vecchi di una settimana semplicemente sfruttando la funzione \textit{scroll} della pagina di \textit{Twitter}: facendo \textit{scroll} verso il fondo pagina è possibile ottenere, tramite chiamate successive ad un \textit{provider JSON}, \textit{tweets} (relativi all'\textit{hashtag} che si sta cercando) via via più vecchi, evitando di incorrere a limitazioni temporali. Tale libreria mette a disposizione un gran numero di criteri di ricerca, utilizzati poi come parametri dell'indirizzo \textit{http}. Nel caso in esame sono stati utilizzati i seguenti parametri:
\begin{itemize}
\item \textit{Since}: una data limite inferiore per limitare la ricerca;
\item \textit{Until}: una data limite superiore per limitare la ricerca;
\item \textit{QuerySearch}: il testo di \textit{query} desiderato. Nel caso in esame, come \textit{query} viene sempre specificato un \textit{hashtag}, il quale identifica una discussione, e vengono considerati tutti e soli i \textit{tweets} creati nell'intervallo temporale specificato e che recano tale \textit{hashtag}.
\end{itemize}
Una volta costruito un oggetto \textit{tweetCriteria}, specificando le informazioni sopra elencate, esso viene passato come parametro al metodo \textit{getTweets} della classe \textit{TweetManager}, il quale si occupa di recuperare tutti i \textit{tweets} che soddisfano i criteri di ricerca. In particolare questo metodo, una volta costruita la \textit{url} contenente tutti i parametri di ricerca specificati, acquisisce la pagina \textit{web} contenente tutti i \textit{tweets} che soddisfano i criteri e converte il risultato in un formato \textit{JSON}. Le informazioni che, nel caso specifico dell'implementazione proposta, vengono estratte dai \textit{tweets} risultanti sono due:
\begin{itemize}
\item \textit{ID} del \textit{tweet};
\item \textit{Username} dell'autore del \textit{tweet}.
\end{itemize}
Gli \textit{ID} dei \textit{tweets} verranno utilizzati per recuperare tutti i \textit{retweets} associati, i quali sono indispensabili per costruire il \textit{retweet graph}.

\subsection{Processo di raccolta dati}
Descritte le caratteristiche e le peculiarità degli strumenti utilizzati per effettuare la raccolta dei dati, ora occorre analizzare il processo che permette di acquisirli e di renderli persistenti. Con tale obiettivo è stata implementata la classe \textit{TwittersRetweets}. Essa fornisce metodi per: 
\begin{enumerate}
\item Specificare i parametri di ricerca (i.e. \textit{Since,Until,QuerySearch}); 
\item Recuperare dalla \textit{social network di Twitter} tutti i \textit{tweets} che soddisfano i parametri di ricerca specificati al punto \textit{1.}, insieme agli utenti che li hanno prodotti; 
\item Recuperare tutti i \textit{retweets} che sono stati prodotti verso i \textit{tweets} recuperati al punto \textit{2.};
\item Organizzare i dati ottenuti in un \textit{file} e renderli persistenti.
\end{enumerate}
Un oggetto della classe \textit{TwittersRetweets} ha pertanto i seguenti attributi:
\begin{itemize}
\item \textit{since}, ossia la data di inizio ricerca; 
\item \textit{until}, ossia la data di fine ricerca;
\item \textit{query}, utilizzato, nel caso in esame, per specificare un \textit{hashtag};
\item \textit{twittapi}, ossia l'oggetto \textit{api}, senza il quale è impossibile eseguire lo \textit{streaming}, ottenuto a seguito dell'autenticazione presso un \textit{server} di \textit{Twitter} mediante l'utilizzo della libreria \textit{Tweepy}.
\end{itemize}
Il processo di raccolta dati viene eseguito mediante l'invocazione del metodo \textit{computeRetweets(path)} su un oggetto della classe \textit{TwittersRetweets} che ha come attributi proprio i parametri di ricerca (i.e. \textit{since, until, query}) e l'oggetto \textit{twittapi}.
L'esecuzione del metodo \textit{computeRetweets(path)} si articola in tre fasi, come è possibile evincere dalla figura \ref{fig:raccolta_dati}. Più precisamente:
\begin{enumerate}
\item Il metodo \textit{computeRetweets(path)} innanzittutto si occupa di recuperare tutti i \textit{tweets} che soddisfano i parametri di ricerca e gli utenti che li hanno creati. Per fare questo, viste le limitazioni temporali che impone l'\textit{Api} ufficiale di \textit{Twitter}, si rende necessario l'utilizzo della libreria \textit{GetOldTweets}: vengono specificati i criteri di ricerca dei \textit{tweets} da recuperare ed infine il \textit{TweetManager} si occupa di cercarli nella pagina \textit{html} di \textit{Twitter}, per poi restituirli. I \textit{tweets} restituiti sono caratterizzati da due importanti parametri: \textit{tweetid}, ossia l'identificativo univoco del \textit{tweet}, e lo \textit{username}, ossia l'utente che l'ha emesso. Per mezzo di questi parametri, in questa fase vengono costruiti due oggetti, ossia:
\begin{itemize}
\item \textit{dictioTwitters}: un dizionario che ha come chiavi gli \textit{usernames} degli utenti che hanno emesso i \textit{tweets} recuperati e come valori dei dizionari della forma \textit{\{'tweetcount' : x\}}, dove \textit{x} è il numero dei \textit{tweets} recuperati (e che quindi soddisfano i criteri di ricerca) che sono attribuibili ad un certo \textit{username};
\item \textit{tweetids}: una lista che ha come elementi dei dizionari della forma \textit{\{tweetid : tweetuser\}}, dove \textit{tweetid} identifica un certo \textit{tweet} e \textit{tweetuser} identifica l'utente che l'ha emesso.
\end{itemize}
Questi oggetti vengono opportunamente popolati e poi forniti come \textit{input} alla fase successiva.
\item La fase \textit{2} si occupa di scansionare la lista \textit{tweetids}, restituita dalla fase precedente, per individuare tutti i \textit{retweets} emessi nei confronti dei \textit{tweets} che soddisfano i criteri di ricerca. A fine scansione, questa fase restituisce il dizionario \textit{dictioRetweets}, che ha come chiavi delle tuple del tipo \textit{(retweetuser,tweetuser)}, dove \textit{retweetuser} è lo \textit{username} di un utente che ha \textit{retweettato} almeno un \textit{tweet} emesso dall'utente identificato da \textit{tweetuser}, e come valori dei dizionari del tipo \textit{\{'retweetcount' : x\}}, dove \textit{x} è il numero di volte che l'utente \textit{retweetuser} ha \textit{retweettato} dei contenuti dell'utente \textit{tweetuser}. 
\\Questa volta, tuttavia, tali \textit{retweets} sono ottenuti per mezzo dell'\textit{Api} ufficiale di \textit{Twitter}. In particolare, per ogni elemento \textit{\{tweetid : tweetuser\}} della lista \textit{tweetids}:
\begin{enumerate}
\item mediante l'invocazione del metodo \textit{retweets}, messo a disposizione dalla \textit{Twitter Api}, vengono recuperati tutti i \textit{retweets} emessi nei confronti del \textit{tweet} identificato dal \textit{tweetid}. Tali \textit{retweets} vengono restituiti sotto forma di una lista di \textit{status objects}, un formato particolare che viene gestito dalla \textit{Twitter Api};
\item per ogni \textit{status object}, che corrisponde ad un particolare \textit{retweet}, viene recuperato il \textit{JSON} che lo descrive, da cui viene estratto lo \textit{username} dell'utente che ha effettuato il \textit{retweet} stesso (accedendo opportunamente ai campi del \textit{JSON}), che chiamiamo \textit{retweetuser}. Se la tupla \textit{(retweetuser,tweetuser)} esiste già come chiave nel dizionario \textit{dictioRetweets} allora viene semplicemente aggiornato il valore del campo \textit{'retweetcount'} corrispondente, altrimenti tale tupla viene aggiunta al dizionario come sua nuova chiave con valore \textit{\{'retweetcount' : 1\}}. 
\\Infine, se lo \textit{username} \textit{retweetuser} non è presente come chiave nel dizionario \textit{dictioTwitters}, viene aggiunta la nuova chiave \textit{retweetuser} con valore \textit{\{'tweetcount' : 0\}}, in quanto \textit{retweetuser} non ha mai emesso \textit{tweets} contenenti l'\textit{hashtag} specificato;
\end{enumerate}
;
\item .
\end{enumerate}

\begin{figure}
\begin{center}
\includegraphics[scale=0.6]{images/raccolta_dati_tesi.png}
\end{center}
\caption{Processo di raccolta dati.}
\label{fig:raccolta_dati}
\end{figure}


\section{Implementazione}

\chapter{Test dell’\textit{edge-recommendation system} in modalità \textit{greedy e non}}
\label{chap:test}

Il sistema implementato è stato sottoposto a vari test con lo scopo di valutare i due algoritmi di \textit{k-edge recommendation} alternativi, ovvero \textit{greedy e non greedy}, ed in modo tale da poterli confrontare tra loro in termini di:
\begin{itemize} 
\item decremento totale dell'\textit{RWC} che ciascuno di essi consente di apportare ad un certo \textit{retweet graph} in input, a parità di numero di archi proposti \textit{k};
\item qualità dei \textit{k} archi scelti, in termini del \textit{$\delta RWC$} associato a ciascuno di essi;
\item tempi di esecuzione.
\end{itemize}

I \textit{retweet graphs} utilizzati come input dei test corrispondono alle discussioni attorno agli \textit{hashtags} controversi \textit{\#beefban, \#indiana, \#russia\_march}, le cui informazioni (i.e. \textit{tweets} e \textit{retweets} emessi nel periodo di osservazione) sono reperibili presso il \textit{repository} degli autori dell'articolo \cite{garimella:paper}. Pertanto, in questo caso, non è stato necessario eseguire il \textit{processo di raccolta dati} descritto nel capitolo precedente ma, per ciascuno degli \textit{hashtags} appena menzionati, è bastato \textit{parsare} il relativo file dei \textit{retweets} (disponibile nel \textit{repository}) e creare il \textit{retweet graph} corrispondente.
\\I \textit{retweet graphs} creati, relativi agli \textit{hashtags} di cui sopra, hanno le seguenti caratteristiche:
\\\\
\begin{tabular}{l*{6}{c}r}
\textbf{Hashtag}         & \textbf{|V|} & \textbf{|E|}  \\
\hline
\#beefban 		 & 1610 & 1978  \\
\#indiana        	 & 2467 & 3143  \\
\#russia\_march   	 & 2134 & 2951  \\
\end{tabular}
\\\\\\
I parametri del sistema sono stati impostati con i seguenti valori:
\begin{itemize}
\item $\alpha = 0.85$;
\item $k_1 = 20$; 
\item $k_2 = 20$;
\item $k = 50$.
\end{itemize}
Nel prossimo paragrafo, per ciascuno dei tre \textit{retweet graphs} in input, verranno mostrati e commentati i risultati dei test relativi alla discesa dell'\textit{RWC} ottenuta a seguito dell'applicazione di ciascuno dei due algoritmi di \textit{recommendation} proposti. 

\section{Discesa dell'\textit{RWC}}

Di seguito inseriamo i grafici che mostrano la discesa dell'\textit{RWC} dei \textit{retweet graphs} relativi agli \textit{hashtags} considerati, nell'ordine:
\begin{enumerate}
\item \textit{\#beefban};
\item \textit{\#indiana};
\item \textit{\#russia\_march}.
\end{enumerate}

\includepdf{images/BEEFBAN_IN_DEG_PROBABILITY_FREE_RWC_DESCENT.pdf}
\includepdf{images/INDIANA_IN_DEG_PROBABILITY_FREE_RWC_DESCENT.pdf}
\includepdf{images/RUSSIA_MARCH_IN_DEG_PROBABILITY_FREE_RWC_DESCENT.pdf}
Ciascuno dei tre grafici, uno per \textit{retweet graph}, ha l'obiettivo di porre a confronto i decrementi dell'\textit{RWC} che rispettivamente i due algoritmi di \textit{edge recommendation} consentono di raggiungere a parità di archi proposti. 
\\In particolare, fissato un \textit{retweet graph g}, il grafico corrispondente mostra due funzioni, rispettivamente di colore \textit{rosso} e di colore \textit{blu}, con valori nel dominio \textit{$0 < j \leq k$}:
\begin{itemize}
\item \textit{RWC(g,j)\textsubscript{greedy}}, ovvero l'\textit{RWC} che caratterizzerebbe il \textit{retweet graph g} qualora i primi \textit{j} archi proposti dall'algoritmo \textit{greedy} si materializzassero nel grafo; 
\item \textit{RWC(g,j)\textsubscript{non-greedy}}, ovvero l'\textit{RWC} che caratterizzerebbe il \textit{retweet graph g} qualora i primi \textit{j} archi proposti dall'algoritmo \textit{non-greedy} si materializzassero nel grafo.
\end{itemize}
Come si nota immediatamente dai grafici, per ogni \textit{retweet graph g} considerato vale:
\\\\
$\textit{RWC(g,j)\textsubscript{greedy}} \leq \textit{RWC(g,j)\textsubscript{non-greedy}}, \forall j = 1,..,k$
\\\\
Ovvero, a parità di archi proposti, l'algoritmo \textit{greedy} consente \textit{sempre} di raggiungere un decremento dell'\textit{RWC} maggiore o uguale, per ciascun grafo \textit{g}. Questa maggiore \textit{efficacia} dell'algoritmo \textit{greedy} non soprende, viste la considerazioni e le analisi condotte nei capitoli precedenti, e deriva sostanzialmente dalla maggiore \textit{qualità}, in termini di decremento del \textit{grado di controversia}, di ciascun arco che propone. 
\\Il paragrafo a seguire si occuperà proprio di confrontare gli archi proposti dai due algoritmi \textit{greedy} e \textit{non-greedy}, in termini dei \textit{$\delta RWC$} corrispondenti.

\section{Qualità degli archi proposti}
Ognuno dei seguenti grafici, corrispondenti ai tre \textit{retweet graphs} considerati, associa ad ogni arco, proposto da ognuno dei due algoritmi, il relativo \textit{$\delta RWC$}.

\includepdf{images/BEEFBAN_IN_DEG_PROBABILITY_FREE_PER_EDGE_DELTA.pdf}
\includepdf{images/INDIANA_IN_DEG_PROBABILITY_FREE_PER_EDGE_DELTA.pdf}
\includepdf{images/RUSSIA_MARCH_IN_DEG_PROBABILITY_FREE_PER_EDGE_DELTA.pdf}
Questa volta ciascuno dei tre grafici, uno per \textit{retweet graph}, ha l'obiettivo di porre a confronto i \textit{$\delta RWC$} associati ai \textit{k} archi che rispettivamente i due algoritmi di \textit{edge recommendation} propongono.
\\Più precisamente, fissato un \textit{retweet graph g}, il grafico corrispondente mostra due funzioni, rispettivamente di colore \textit{rosso} e di colore \textit{blu}, con valori nel dominio \textit{$0 < i \leq k$}\footnote{Ovvero l'\textit{i-esimo} arco proposto.}:
\begin{itemize}
\item \textit{$\delta RWC(g,i)\textsubscript{greedy}$}, ovvero il \textit{$\delta RWC$} che l'\textit{i-esimo} arco proposto dall'algoritmo \textit{greedy} consentirebbe di ottenere qualora si materializzasse successivamente a \textit{tutti} gli archi proposti con indici da \textit{1} a \textit{i-1};
\item \textit{$\delta RWC(g,i)\textsubscript{non-greedy}$}\footnote{Ciascun \textit{$\delta RWC(g,i)\textsubscript{non-greedy}$} non è il \textit{$\delta RWC$} utilizzato dall'algoritmo \textit{non-greedy} come criterio di scelta dell'\textit{i-esimo} arco ma è l'effettivo decremento dell'\textit{RWC} che l'\textit{i-esimo} arco proposto apporterebbe se si materializzasse secondo l'ordine di scelta.}, ovvero il \textit{$\delta RWC$} che l'\textit{i-esimo} arco proposto dall'algoritmo \textit{non-greedy} consentirebbe di ottenere qualora si materializzasse successivamente a \textit{tutti} gli archi proposti con indici da \textit{1} a \textit{i-1}.
\end{itemize}
Anche in questo caso è evidente che, per ogni \textit{retweet graph g} considerato e $\forall i = 1,..,k$, vale:
\\\\
$\textit{$\delta RWC(g,i)\textsubscript{greedy}$} \leq \textit{$\delta RWC(g,i)\textsubscript{non-greedy}$}$
\\\\
Ovvero, nell'ipotesi che tutti gli archi precedentemente proposti vengano accettati, il \textit{$\delta RWC$} associato all'\textit{i-esimo} arco proposto dall'algoritmo \textit{greedy} è minore o uguale al \textit{$\delta RWC$} associato all'\textit{i-esimo} arco proposto dall'algoritmo \textit{non-greedy}, $\forall i = 1,..,k$. Quest'osservazione implica che, generalmente, gli archi scelti da \textit{greedy} sono migliori \textit{qualitativamente} rispetto a quelli scelti da \textit{non-greedy}, poiché determinano un maggior decremento dell'\textit{RWC} associato al \textit{retweet graph} in input.
\\I risultati sinora ottenuti derivano senz'altro dalla maggior precisione del criterio di scelta degli archi dell'algoritmo \textit{greedy} rispetto a quello dell'algoritmo \textit{non-greedy}. 
\\A tal proposito, è possibile utilizzare il \textit{tool di visualizzazione} introdotto nel capitolo precedente per analizzare, per ogni \textit{retweet graph} in input, le caratteristiche dei \textit{k} archi scelti da ciascuno dei due algoritmi e dei nodi coinvolti: quest'analisi potrebbe chiarire ulteriormente le cause che rendono un algoritmo più efficace dell'altro. 
\\A fine capitolo inseriamo gli \textit{outputs} del \textit{tool}, ciascuno dei quali descrive i risultati dell'applicazione di uno dei due algoritmi di \textit{edge recommendation} su un \textit{retweet graph} in input.
\\
Con riferimento alle figure da \ref{fig:beefgreedy} a \ref{fig:russianotgreedy}, si nota che generalmente i \textit{k} archi scelti dall'algoritmo \textit{non greedy} tendono a formare una struttura a \textit{stella}: infatti la maggior parte di essi condivide uno stesso nodo \textit{endpoint}. Ciò è dovuto al fatto che l'algoritmo \textit{non greedy} utilizza come metrica di scelta degli archi il \textit{$\delta RWC$} che ciascuno di essi apporterebbe se fosse aggiunto \textit{individualmente} al grafo, ignorando la potenziale diminuzione dell'efficacia individuale causata dalla loro interazione reciproca. In generale, si osserva che maggiore è la tendenza degli archi scelti a condividere uno stesso \textit{endpoint} e minore è il decremento dell'\textit{RWC} che collettivamente riescono ad apportare: questo è il caso degli archi proposti dall'algoritmo \textit{non greedy} e determina il suo \textit{deficit} di efficacia. 
\section{Tempi di esecuzione}
Questo paragrafo ha lo scopo di mostrare e commentare i tempi di esecuzione dei due algoritmi di \textit{k-edge recommendation}, espressi in funzione del \textit{retweet graph} in input. Si consideri la seguente tabella:
\\\\
\begin{tabular}{l*{6}{c}r}
\textbf{Hashtag}         & \textbf{greedy} & \textbf{non greedy}  \\
\hline
\#beefban 		 & 7252 sec &  149 sec  \\
\#indiana        	 & 17580 sec & 352 sec \\
\#russia\_march   	 & 12720 sec & 253 sec  \\
\end{tabular}
\\\\\\
Come era facile intuire, i tempi di esecuzione di entrambi gli algoritmi crescono all'aumentare della complessità del \textit{retweet graph} in input, complessità espressa in termini del numero di nodi e del numero di archi. Addirittura l'algoritmo \textit{greedy} con input il \textit{retweet graph \#indiana}, che è il grafo più ingente tra i tre esaminati, necessita di quasi 5 ore di esecuzione.  
\\Il tempo di esecuzione è anche funzione dell'algoritmo di \textit{k-edge recommendation} utilizzato: infatti i test hanno confermato le osservazioni riportate nel capitolo 2, nelle quali si asseriva che, a parità di valori dei parametri di sistema ($\alpha, k_1, k_2, k$) e del \textit{retweet graph} in input, l'algoritmo \textit{greedy} è circa \textit{k} volte più lento dell'algoritmo \textit{non-greedy}. Nei casi in esame:
\begin{itemize}
\item 7252 sec $\simeq$  \textit{k} $\times$ 149 sec;
\item 17580 sec $\simeq$ \textit{k} $\times$ 352 sec;
\item 12720 sec $\simeq$ \textit{k} $\times$ 253 sec.
\end{itemize}
Dove \textit{k} è il numero di archi da proporre ed in questo caso vale 50.
\\La maggiore precisione e, quindi, la maggiore efficacia dell'algoritmo \textit{greedy} si paga con il suo tempo di esecuzione, il quale può risultare addirittura proibitivo nel caso di grafi molto ingenti. Pertanto, la scelta dell'algoritmo da utilizzare va effettuata caso per caso e dipende dalle esigenze che si vogliono soddisfare, che riguardino i tempi di esecuzione o l'efficacia degli archi proposti.

\begin{figure}
\begin{center}
\includegraphics[scale=0.5]{images/beefban_in_degree_greedy_probability_free.png}
\end{center}
\caption{Porzione dell'\textit{output} del \textit{tool} a seguito dell'esecuzione di \textit{greedy} sul \textit{retweet graph \#beefban}.}
\label{fig:beefgreedy}
\end{figure}

\begin{figure}
\begin{center}
\includegraphics[scale=0.5]{images/beefban_in_degree_probability_free.png}
\end{center}
\caption{Porzione dell'\textit{output} del \textit{tool} a seguito dell'esecuzione di \textit{non greedy} sul \textit{retweet graph \#beefban}.}
\label{fig:beefnotgreedy}
\end{figure}

\begin{figure}
\begin{center}
\includegraphics[scale=0.5]{images/indiana_in_degree_greedy_probability_free.png}
\end{center}
\caption{Porzione dell'\textit{output} del \textit{tool} a seguito dell'esecuzione di \textit{greedy} sul \textit{retweet graph \#indiana}.}
\label{fig:indigreedy}
\end{figure}

\begin{figure}
\begin{center}
\includegraphics[scale=0.5]{images/indiana_in_degree_probability_free.png}
\end{center}
\caption{Porzione dell'\textit{output} del \textit{tool} a seguito dell'esecuzione di \textit{non greedy} sul \textit{retweet graph \#indiana}.}
\label{fig:indinotgreedy}
\end{figure}

\begin{figure}
\begin{center}
\includegraphics[scale=0.5]{images/russia_march_in_degree_greedy_probability_free.png}
\end{center}
\caption{Porzione dell'\textit{output} del \textit{tool} a seguito dell'esecuzione di \textit{greedy} sul \textit{retweet graph \#russia\_march}.}
\label{fig:russiagreedy}
\end{figure}

\begin{figure}
\begin{center}
\includegraphics[scale=0.5]{images/russia_march_in_degree_probability_free.png}
\end{center}
\caption{Porzione dell'\textit{output} del \textit{tool} a seguito dell'esecuzione di \textit{non greedy} sul \textit{retweet graph \#russia\_march}.}
\label{fig:russianotgreedy}
\end{figure}




\chapter{Conclusioni e sviluppi futuri}
\label{chap:conclusioni}



\appendix
% INCLUSIONE APPENDICI - - PERSONALIZZARE - TENERE COERENTE CON LISTA IN ALTO
\include{app_a}
\chapter{Another appendix}
\label{app:b}
\Blindtext

%%%%%%%%%%%%%%%%%%%%%%%%%%%%%%%%%%%%%%%%%%%%%%%%%%%%%%%%%%%%%%%

% BIBLIOGRAFIA
\addcontentsline{toc}{chapter}{\refname}
\nocite{*}
\printbibliography

\end{document}
