%CLASSE DOCUMENTO - LINGUA E DIMENSIONE FONT
\documentclass[trieste]{toptesi}

%%%%%%%%%%%%%%%%%%%%%%%%%%%%%%%%%%%%%%%%%%%%%%%%%%%%%%%%%%%%%%%

% INCLUSIONE PACCHETTI
%\usepackage[utf8x]{inputenc}
\usepackage[utf8]{inputenc} %utf8
\usepackage[italian]{babel}
\usepackage[T1]{fontenc}
\usepackage{blindtext}
\usepackage{graphicx,wrapfig}
\usepackage{booktabs}
\usepackage{lmodern}
\usepackage{varioref}
\usepackage{url}
\usepackage{array}
\usepackage{paralist}{\obeyspaces\global\let =\space}
\usepackage{verbatim} 
\usepackage{subfig}
\usepackage{tabularx}
\usepackage{amsmath}
\usepackage{amsfonts}
\usepackage{float}
\usepackage{amssymb}
\usepackage{multicol}
\usepackage{multirow}
\usepackage{listings}
\usepackage[pass]{geometry}
\usepackage[figuresright]{rotating}
\usepackage{algorithm}
\usepackage{algorithmic}
\usepackage{amsmath}
\usepackage[babel]{csquotes}
\usepackage{hyperref}
\usepackage[backend=bibtex]{biblatex}

%%%%%%%%%%%%%%%%%%%%%%%%%%%%%%%%%%%%%%%%%%%%%%%%%%%%%%%%%%%%%%%

% CONFIGURAZIONE LINK E RIFERIMENTI
\hypersetup{%
    pdfpagemode={UseOutlines},
    bookmarksopen,
    pdfstartview={FitH},
    colorlinks,
    linkcolor={black}, %COLORE DEI RIFERIMENTI AL TESTO
    citecolor={blue}, %COLORE DEI RIFERIMENTI ALLE CITAZIONI
    urlcolor={blue} %COLORI DEGLI URL
}

%%%%%%%%%%%%%%%%%%%%%%%%%%%%%%%%%%%%%%%%%%%%%%%%%%%%%%%%%%%%%%%

% CONFIGURAZIONE LISTATI/CODICE - CANCELLARE SE NON NECESSARIO
% PYTHON - BIANCO E NERO
\lstset{%
	captionpos=b,
	language=Python,
	basicstyle =\small\ttfamily,
	keywordstyle=\color{black}\bfseries,
	breaklines=true,
	breakatwhitespace=true,
	frame=lines,
	numbers=left,
	numberstyle=\footnotesize,
}

%%%%%%%%%%%%%%%%%%%%%%%%%%%%%%%%%%%%%%%%%%%%%%%%%%%%%%%%%%%%%%%

% FRENCHSPACING VA _SEMPRE_ ABILITATO PER DOCUMENTI IN ITALIANO
\frenchspacing

%%%%%%%%%%%%%%%%%%%%%%%%%%%%%%%%%%%%%%%%%%%%%%%%%%%%%%%%%%%%%%%

%DEFINIZIONE SEZIONI IN NUMERAZIONE ROMANA
%ELENCO DEI LISTATI/CODICI
\makeatletter
\newcommand\listofcodes{%
 \iffrontmatter\else\frontmattertrue\fi
 \if@openright\cleardoublepage\else\clearpage\fi
 % change the meaning of \chapter in a group
 \begingroup\def\chapter##1{\@schapter}
 \phantomsection % for the hyperlink
 \lstlistoflistings 
 \endgroup
} 
\makeatother

%%%%%%%%%%%%%%%%%%%%%%%%%%%%%%%%%%%%%%%%%%%%%%%%%%%%%%%%%%%%%%%

% INFORMAZIONI PDF - PERSONALIZZARE
\pdfinfo{%
  /Title    (Riduzione del grado di controversia delle rete sociali connettendo punti di vista opposti)
  /Author   (Stefano Agostini)
  /Subject  (Non definito)
  /Keywords (Controversia)
}

%%%%%%%%%%%%%%%%%%%%%%%%%%%%%%%%%%%%%%%%%%%%%%%%%%%%%%%%%%%%%%%

\ateneo{Università degli studi di Roma Tor Vergata}

% LOGO UNIVERSITA
\logosede{images/Logo-Uni-Tor-Vergata.png}

\facolta{Ingegneria}

\corsodilaurea{Ingegneria Informatica}

% TIPOLOGIA TESI
\TesiDiLaurea{Tesi di Laurea Magistrale}

% TITOLO
\titolo{Riduzione del grado di controversia delle rete sociali connettendo punti di vista opposti}

% RELATORE/I
\relatore{Giuseppe F. Italiano}
\secondorelatore{Nikos Parotsidis}

% CANDIDATO - DICITURA (MANTENERE I DUE PUNTI) - CANCELLARE O DECOMMENTARE
%\CandidateName{Candidate:}

% CANDIDATO - NOME E COGNOME
\candidato{\tabular{l}Stefano \textsc{Agostini}\\matricola: 0234240\endtabular}

% DATA - MESE ANNO
\sedutadilaurea{A.A. 2017/2018}

%%%%%%%%%%%%%%%%%%%%%%%%%%%%%%%%%%%%%%%%%%%%%%%%%%%%%%%%%%%%%%%

% LISTA DEI CAPITOLI DA INCLUDERE - PERSONALIZZARE
\includeonly{%
chap_0,%
chap_1,%
chap_2,%
app_a,%
app_b,%
}

% FILE DI BIBLIOGRAFIA
\bibliography{bibliography} 


%%%%%%%%%%%%%%%%%%%%%%%%%%%%%%%%%%%%%%%%%%%%%%%%%%%%%%%%%%%%%%%

% INIZIO DOCUMENTO
\begin{document}

\frontespizio

%%%%%%%%%%%%%%%%%%%%%%%%%%%%%%%%%%%%%%%%%%%%%%%%%%%%%%%%%%%%%%%

%INTERLINEA - DEFAULT 1 - NON ESAGERATE, NON SUPERATE MAI 1.3 ;)
\interlinea{1.3}

%%%%%%%%%%%%%%%%%%%%%%%%%%%%%%%%%%%%%%%%%%%%%%%%%%%%%%%%%%%%%%%

\frontmatter

% DEDICA - PERSONALIZZARE
% VSPACE - PROPORZIONE USATA PER CENTRATURA VERTICALE DEL TESTO
% FLUSHRIGHT - ALLINEAMENTO ORIZZONTALE A DESTRA
\vspace*{\stretch{1}}
\begin{flushright}
\noindent
Da inserire qui tutte le dediche della mia tesi...
\end{flushright}
\vspace*{\stretch{6}}
\cleardoublepage

%%%%%%%%%%%%%%%%%%%%%%%%%%%%%%%%%%%%%%%%%%%%%%%%%%%%%%%%%%%%%%%

% ABSTRACT - PERSONALIZZARE
\sommario
Esistono, e sono sempre esistiti, particolari temi, detti \textit{controversi}, per i quali ognuno di noi si schiera come sostenitore o come oppositore. L'effetto quasi immediato è la divisione della popolazione in due gruppi che hanno visioni opposte sull'argomento \textit{controverso} in considerazione e che difficilmente scambiano tra loro informazioni e punti di vista, non facendo altro che acutizzare la loro inconciliabilità di opinione: gruppi di individui di questo genere sono anche detti \textit{echo chambers}. Gli invidui facenti parte della stessa \textit{echo chamber} hanno quindi le stesse credenze e lo stesso parere riguardo all'argomento \textit{controverso}: essi rinforzano a vicenda le proprie opinioni e non sono esposti a punti di vista opposti ai propri (i.e. le opinioni che caratterizzano l'altra \textit{echo chamber}). Non è immune da tale fenomeno il mondo dei \textit{social media}. In particolare, \textit{Twitter} identifica un particolare \textit{topic} mediante un \textit{hashtag} e la discussione riguardo a tale \textit{topic} (e.g.\textit{\#novax}) viene descritta mediante un \textit{endorsement graph}, ovvero un \textit{grafo diretto} in cui ciascun nodo rappresenta un utente che partecipa alla discussione e vi è un arco diretto da un nodo \textit{x} ad un nodo \textit{y} se e solo se l'utente \textit{x} approva l'opinione dell'utente \textit{y}. Quando il \textit{topic} in considerazione è \textit{controverso}, la struttura dell'\textit{endorsement graph} mette in luce la presenza di gruppi di utenti molto connessi al loro interno ma che comunicano poco tra loro: la distanza di opinione di tali gruppi viene quantificata mediante il così detto \textit{indice di controversia} del grafo, che viene misurato, nel lavoro di tesi proposto, da una metrica basata sui \textit{random walks}. \textit{Endorsement graphs} con elevati \textit{indici di controversia} sono costituiti da \textit{echo chambers} molto divise tra loro, in ciascuna delle quali è amplificata una visione univoca ed acritica sull'argomento \textit{controverso} considerato. \\L'obiettivo di questa tesi è quello di implementare un \textit{edge-recommendation system} il quale, dopo aver analizzato un dato \textit{endorsement graph} di un particolare \textit{topic controverso} presente nella \textit{social network} di \textit{Twitter} ed averne individuato le \textit{echo chambers}, permetta di individuare i \textit{k} archi diretti (\textit{k} fornito in \textit{input}) che, se aggiunti al grafo, producono il più alto decremento del \textit{grado di controversia}. Poiché gli \textit{endorsement graph} di questo tipo sono generalmente costituiti da un numero molto elevato di nodi (indicato con \textit{n}), risolvere tale problema di ottimizzazione considerando tutte le possibili combinazioni di archi (\textit{O}${n^2\choose k}$) è evidentemente molto costoso dal punto di vista computazionale e molto inefficiente anche per quanto riguarda i tempi. Pertanto l'approccio seguito è quello adottato nell'articolo \cite{famous:paper}, ossia considerare solo un sottoinsieme degli archi possibili ed estrarre da questo sottoinsieme i \textit{k} più promettenti. Chiaramente questa soluzione restituisce archi meno efficaci per quanto riguarda la riduzione del \textit{grado di controversia} rispetto a quelli proposti dall'approccio \textit{brute force} ma è molto più efficiente; in particolare, la scelta dell'euristica che permette di definire il sottoinsieme degli archi da considerare è cruciale. In particolare l'euristica utilizzata nell'articolo \cite{famous:paper}
consiste nel considerare solo gli archi tra vertici di grado alto di ciascuna comunità (o \textit{echo chamber}). 








%%%%%%%%%%%%%%%%%%%%%%%%%%%%%%%%%%%%%%%%%%%%%%%%%%%%%%%%%%%%%%%

% INDICI - ELIMINARE GLI INDICI NON NECESSARI

% INDICE GENERALE
\tableofcontents

% INDICE DELLE FIGURE
\listoffigures

% INDICE DELLE TABELLE
\listoftables

% INDICE DEI CODICI
\listofcodes

%%%%%%%%%%%%%%%%%%%%%%%%%%%%%%%%%%%%%%%%%%%%%%%%%%%%%%%%%%%%%%%

\mainmatter

% INCLUSIONE FILE CAPITOLI - PERSONALIZZARE - TENERE COERENTE CON LISTA IN ALTO
\chapter{Introduzione}
\label{chap:introduzione}

Il fenomeno della \textit{polarizzazione} degli utenti attorno a \textit{topic controversi} che si propongono nei \textit{social media} è ben noto ed il suo studio è già stato affrontato in alcuni articoli scientifici (tra cui \cite{musco:paper}\cite{morales:paper}). Potremmo definire la \textit{polarizzazione} come segue:
\\\\
\textit{Situazione che determina la divisione della popolazione in gruppi con punti di vista opposti riguardo ad un certo argomento.}
\\\\Molto spesso i \textit{social media}, mediante algoritmi di \textit{recommendation}, espongono gli utenti solo a contenuti che si addicono e sono conformi alle loro opinioni e, pertanto, non fanno altro che aggravare il loro stato di polarizzazione. Tutto ciò determina la formazione delle così detto \textit{echo-chambers}, ossia:
\\\\
\textit{Situazioni in cui individui che hanno lo stesso parere su un certo argomento rafforzano l'opinione reciproca ma non vengono esposti ad opinioni opposte alla propria.}
\\\\
Lo scopo di questa tesi è quello di sviluppare un sistema in grado far comunicare, nel modo più efficace possibile, queste \textit{echo-chambers} così da esporre gli individui a punti di vista opposti ai propri e ridurre la \textit{controversia} della discussione. L'ambiente in cui opera il \textit{framework} proposto è il \textit{social network} di \textit{Twitter}, in cui gli argomenti delle discussioni vengono identificati da \textit{hashtags}, i contenuti relativi vengono espressi dagli utenti attraverso i \textit{tweets} e le condivisioni di opinione attraverso i \textit{retweets}. Una discussione nell'ambiente di \textit{Twitter} può essere descritta mediante un \textit{endorsement graph}, ossia un grafo i cui nodi sono utenti che hanno espresso almeno un'opinione mediante un \textit{tweet} ed i cui archi rappresentano i \textit{retweets}\footnote{Vi è un arco da un nodo \textit{u} ad un nodo \textit{v} se e solo se l'utente \textit{u} ha \textit{retweettato} almeno un \textit{tweet} di \textit{v}.}.\\Il \textit{framework} acquisisce da \textit{Twitter} i dati necessari per elaborare \textit{endorsement graphs} di \textit{topic controversi}, ne analizza la struttura estraendone le \textit{echo chambers} ed infine implementa un \textit{edge-recommendation system} che ha lo scopo di ridurre il \textit{grado di controversia} creando connessioni (\textit{bridges}) tra utenti che hanno punti di vista opposti (appartengono a \textit{echo chambers} distinte). Infatti la naturale propensione degli individui a dare credito solo a notizie e contenuti che si addicono al proprio parere fa sì che, in assenza di un intervento esterno di \textit{edge-recommendation}, essi rafforzino sempre più la propria convinzione, anche qualora questa fosse sbagliata o acritica.\\\\Il sistema proposto in questo lavoro di tesi si compone di più fasi successive, attraverso le quali raggiunge l'obiettivo preposto; nel seguito vengono descritte sinteticamente tali fasi, fornendo le nozioni teoriche e le assunzioni sulle quali si fonda la loro implementazione.
\\\\La prima fase si occupa dell'acquisizione dei dati da \textit{Twitter} e quindi della costruzione dell'\textit{endorsement graph} associato. Il \textit{framework} utilizza la libreria \textit{Python Tweepy} (che accede all'\textit{API} di \textit{Twitter}) per ottenere tutti i \textit{tweets} e \textit{retweets} emessi dagli utenti, riguardanti l'\textit{hashtag} fornito in input, in un certo intervallo di tempo. Al termine della collezione di tali dati, essi vengono \textit{parsati} per la costruzione dell'\textit{endorsement graph} che descrive la discussione. Gli \textit{endorsement graphs}, nel particolare ambiente di \textit{Twitter}, prendono anche il nome di \textit{retweet graphs}. Dato un \textit{hashtag} controverso, viene a formarsi naturalmente una discussione a riguardo, nella quale gli utenti esprimono una propria opinione e possono approvare il punto di vista delle così dette \textit{autorità}: nel caso particolare di \textit{Twitter} questa \textit{approvazione} si realizza mediante lo strumento del \textit{retweet}, ossia se l'utente \textit{u} fa \textit{retweet} di un \textit{tweet} prodotto dall'utente \textit{v} allora ne approva l'opinione. Ne deriva la formazione di un grafo diretto \textit{G(V,E)} costituito da \textit{n} nodi (gli utenti che partecipano alla discussione) ed i cui archi (\textit{retweets}) esprimono relazioni di condivisione di punti di vista: proprio questo grafo rappresenta l'\textit{output} di questa fase.
\\\\La seconda fase si occupa di rilevare le \textit{echo-chambers} dell'\textit{endorsement graph} in input e di calcolarne il \textit{grado di controversia}. Le \textit{echo-chambers} sono due sottoinsiemi dei nodi del grafo \textit{X,Y}, ben separati tra loro (vi sono pochi archi che li congiungono) e tali che \textit{$X \cup Y = V$} e \textit{$X \cap Y =  \emptyset$}. Tale ripartizione dei nodi può essere ottenuta mediante l'utilizzo di un algoritmo di \textit{graph-partitioning}. Nel lavoro proposto è stato utilizzato l'algoritmo di \textit{Girvan-Newman}: esso è un metodo gerarchico usato per rilevare le comunità in sistemi complessi e la cui esecuzione produce un \textit{dendrogramma}, le cui foglie sono i nodi del grafo. Individuate le \textit{echo-chambers}, questa fase si occupa di quantificare la \textit{controversia} della discussione.
\\Il sistema sviluppato utilizza una metrica basata sul concetto di \textit{random walk} per misurare il \textit{grado di controversia} associato al \textit{topic} analizzato (attorno al quale si svolge la discussione nel \textit{social network} di \textit{Twitter}). Più precisamente, per misurare il \textit{grado di controversia} della rete, viene utilizzata la funzione \textit{Random-walk controversy score}:
\\\\
$\textit{RWC(G,X,Y) = (c\textsubscript{x} - c\textsubscript{y})\textsuperscript{T}(r\textsubscript{x} - r\textsubscript{y})}$\footnote{La definizione del \textit{Random-walk controversy score} è tratta dall'articolo \cite{garimella:paper}.}
\\\\Dove \textit{c\textsubscript{x}} è un vettore di dimensione \textit{n} (numero di vertici dell'\textit{endorsement graph}) che ha valore \textit{1} nelle coordinate corrispondenti ai vertici di grado alto dell'\textit{echo-chamber X} e \textit{0} altrove; similmente viene definito \textit{c\textsubscript{y}}. Infine \textit{r\textsubscript{x}} è il vettore di \textit{PageRank} personalizzato per un \textit{random walk} che parte dai nodi dell'\textit{echo-chamber X}; similmente viene definito \textit{r\textsubscript{y}}. Valori alti di \textit{RWC(G,X,Y)} indicano che, all'equilibrio del \textit{random walk}, è bassa la probabilità di essere nell'\textit{echo-chamber} opposta a quella di partenza: questo è indice di elevata \textit{controversia}. 
\\\\La terza fase ha lo scopo di ridurre il \textit{grado di controversia} rilevato nella fase precedente mediante l'esecuzione di un \textit{edge-recommendation system}. In particolare, il sistema implementato permette, utilizzando in alternativa un algoritmo \textit{greedy} o uno \textit{non-greedy}, di proporre \textit{k} archi (che farebbero da \textit{bridges} tra le due \textit{echo chambers}) in grado di ridurre questo indice; l'\textit{edge-recommendation system} proposto restituisce archi la cui efficacia approssima quella degli archi che sono soluzione del problema di ottimizzazione originario che, come detto precedentemente, ha una complessità di un livello troppo elevato (\textit{O}(${n^2\choose k}$)) per essere risolto in tempi accettabili. Di seguito la definizione del problema di ottimizzazione originario:
\\
\begin{equation*}
\begin{aligned}
& \underset{E\textsubscript{k}}{\text{minimize}}
& & RWC(G(V, E \cup E\textsubscript{k}),X,Y) \\
& \text{subject to}
& & E\textsubscript{k} \subseteq V \times V \setminus E, \left|{E\textsubscript{k}}\right| = k
\end{aligned}
\end{equation*}
\\
Ossia il problema originario consiste nel trovare l'insieme di \textit{k} archi, considerando \textit{tutti} gli archi non ancora presenti nel grafo, che se si materializzassero \textit{minimizzerebbero} l'\textit{indice di controversia}.
L'euristica proposta in questo lavoro permette di restringere il dominio degli archi considerati, consentendo di ottenere risultati paragonabili con quelli ottenibili mediante la soluzione del problema di ottimizzazione appena descritto e con il vantaggio di avere una complessità computazionale di molto inferiore; essa, forniti in \textit{input} i valori di \textit{k\textsubscript{1}} e \textit{k\textsubscript{2}} (interi positivi):
\begin{enumerate}
\item considera i \textit{k\textsubscript{1}} vertici con \textit{in-degree}\footnote{Il numero di archi diretti del grafo che hanno come nodo di destinazione un nodo \textit{x} è detto \textit{in-degree} di \textit{x}.} più alto dell'\textit{echo-chamber X} e i \textit{k\textsubscript{2}} vertici con \textit{in-degree} più alto dell'\textit{echo-chamber Y};
\item costruisce il dominio degli archi considerati come l'insieme di tutti i possibili archi diretti, non presenti ancora nel grafo, che abbiano come estremi un vertice dell'insieme \textit{K\textsubscript{1}} e uno dell'insieme \textit{K\textsubscript{2}};  
\end{enumerate}
Gli algoritmi \textit{greedy} e \textit{non-greedy} operano entrambi a partire dal dominio così definito. Hanno lo scopo di proporre i \textit{k} archi più promettenti del dominio in termini del decremento del grado di controversia \textit{RWC(G,X,Y)} che consentirebbero qualora si materializzassero nel grafo; la differenza sta nella modalità di selezione. In particolare:
\begin{enumerate}
\item l'algoritmo \textit{greedy} sceglie \textit{avidamente} ognuno dei \textit{k} archi: esso impiega \textit{k} passi, in ognuno dei quali sceglie l'arco più appetibile del dominio (ossia l'arco del dominio non ancora presente nel grafo associato al $\delta RWC$ maggiore) e lo aggiunge al grafo.
\item l'algoritmo \textit{non-greedy} ordina una sola volta tutti gli archi del dominio secondo il $\delta RWC$ che ognuno consente e sceglie in un solo passo i \textit{k} archi migliori. Ne deriva una maggiore efficienza nei tempi di esecuzione ma una minore precisione nella scelta degli archi da proporre.
\end{enumerate}
%qui descrivo sinteticamente i due algoritmi
In pratica, per ridurre la \textit{controversia}, si propone ad un certo insieme di utenti il contenuto (i.e. \textit{tweets}) di utenti che hanno posizioni sull'argomento opposte rispetto alle proprie, sperando che la maggior parte di loro possa farne \textit{endorsement} mediante lo strumento del \textit{retweet}: ciò provocherebbe la formazione di nuovi archi tra le due comunità (\textit{echo chambers}) con l'effetto di ridurre il \textit{grado di controversia} dell'intero \textit{retweet graph}.  
\\\\Il \textit{framework} implementato è stato testato su tre \textit{retweet graphs} di tre \textit{topics} (ossia \textit{hashtags}) controversi di \textit{Twitter}: \textit{\#beefban,\#russia\_march,\#indiana}. I test sono stati prodotti seguendo l'approccio del \textit{paper}\cite{garimella:paper}, che ha lo scopo di mostrare il livello di \textit{controversia} del grafo in funzione della quantità di archi al momento aggiunti. Tuttavia l'obiettivo dei test, in questa tesi, è in primo luogo quello di fornire un confronto dell'efficacia dell'\textit{edge-recommendation system} basato sull'algoritmo \textit{greedy} con l'efficacia dell'\textit{edge-recommendation system} basato sull'algoritmo \textit{non-greedy}. Per \textit{efficacia} si intende:
\\\\
\textit{Fissato come obiettivo il decremento del grado di controversia di una quantità $\Delta RWC$, la quantità minima di archi, consigliati dal sistema, che gli utenti devono accettare per poter raggiungerlo.}
\\\\
In secondo luogo i test hanno lo scopo di fornire un raffronto dei tempi di esecuzione dei due algoritmi al variare dell'entità del grafo in input, entità espressa in termini del numero di nodi \textit{n} e del numero di archi \textit{e}.
\\\\Per terminare, il \textit{framework} offre un \textit{tool} per la visualizzazione degli archi proposti, evidenziando le caratteristiche dei nodi coinvolti tra cui l'\textit{in degree} (ossia il grado in ingresso); inoltre è utilizzato un algoritmo di \textit{coloring}\footnote{Con \textit{coloring} si intende una colorazione esatta dei vertici, cioè un'etichettatura dei vertici del grafo con colori tali che nessuna coppia di vertici che condividono lo stesso arco abbiano lo stesso colore.} per classificare tali nodi colorandoli in modo diverso, in funzione dell'\textit{echo-chamber} a cui appartengono.
\\Di seguito è illustrata brevemente l'organizzazione della tesi.
\section{Organizzazione della tesi}
\label{sec:organizzazione}
L'esposizione del lavoro di tesi ha l'obiettivo di fornire \textit{dettagli} riguardanti:
\begin{itemize}
%\item \textit{lo stato dell'arte;}
\item \textit{la teoria che è alla base del problema affrontato;}
\item \textit{la raccolta dei dati e l'implementazione del framework;}
\item \textit{le modalità in cui sono stati effettuati i test ed i risultati ottenuti;}
\item \textit{sviluppi futuri.}
\end{itemize}
%Lo stato dell'arte verrà trattato nell'omonimo capitolo. 
Gli approfondimenti teorici verranno illustrati nel capitolo \textit{Teoria alla base del problema ed algoritmi per la risoluzione}. Il capitolo \textit{Raccolta dati ed implementazione} si occuperà di dare dettagli sulle tecnologie utilizzate per la raccolta dei dati e di illustrare i dettagli implementativi. I risultati sperimentali ottenuti dai test condotti (nelle modalità \textit{greedy e non}) e le osservazioni ad essi riguardanti sono trattati nel capitolo \textit{Test dell'edge-recommendation system in modalità greedy e non}. Il capitolo \textit{Conclusioni e sviluppi futuri} tratterà le conclusioni tratte dai risultati dei test ed approfondirà le sfide ed i propositi di miglioramento del sistema implementato.

%This is a reference to a chapter \ref{chap:quo}. This is a reference to a figure \ref{fig:doge}. This is a reference to some code \ref{lst:hello}. This is a citation %\cite{famous:paper}.

%\lstinputlisting[basicstyle=\fontsize{8}{10}\selectfont\ttfamily ,label=lst:prova, firstline=1, lastline=34, caption={I directly included a portion of a file}]{code/prova.py}

%\begin{lstlisting}[language=Java, label=lst:java, caption={Some code in another language than the default one}]
%public void prepare(AClass foo) {
%        AnotherClass bar = new AnotherClass(foo)
%}
%\end{lstlisting}

%\Blindtext

%\begin{figure}
%\begin{center}
%\includegraphics[width=0.5\columnwidth]{images/doge.png}
%\end{center}
%\caption{This is not a figure. It's a caption.}
%\label{fig:doge}
%\end{figure}

\chapter{Lo stato dell'arte}
\label{chap:stato}
Nonostante il \textit{Web} offra metodi immediati per accedere a qualsiasi tipo di informazione, molti studi (e.g. \textit{Liao et al.}\cite{liao:paper}) dimostrano che, quando un utente può scegliere, esso preferisce essere esposto a contenuti che non contrastano con la sua opinione. Questo fenomeno ha portato ad una maggiore frammentazione e polarizzazione online.

\chapter{Teoria alla base del problema ed algoritmi per la risoluzione}
\label{chap:teoria}
\section{Misura del grado di controversia}
Prima di dare una definizione formale del \textit{random-walk controversy score}, elenchiamo ed illustriamo i passi necessari per calcolarlo.
\begin{enumerate}
\item Fissato il \textit{topic t} per il quale si vuole quantificare il grado di controversia, è possibile descrivere la discussione mediante l'\textit{endorsement graph G(V,E)}. Nell'ambiente di \textit{Twitter}, il \textit{topic t} è identificato da un \textit{hashtag} (e.g.\textit{\#hashtag}) ed i nodi del grafo rappresentano gli utenti che hanno preso parte alla discussione utilizzando almeno una volta tale \textit{hashtag} nei loro \textit{tweets}; gli archi del grafo identificano i \textit{retweets} tra gli utenti, che esprimono relazioni di condivisione di opinione riguardo il \textit{topic}.
\item Ipotizzando che il \textit{topic t} sia controverso, è possibile partizionare i nodi del grafo \textit{G(V,E)} in due insiemi \textit{X},\textit{Y} ben separati tra loro (i.e. vi sono pochi archi che li interconnettono). Tali insiemi quindi soddisfano le seguenti proprietà:
\begin{enumerate}
\item $\textit{X}\cup\textit{Y} = \textit{V}$;
\item $\textit{X}\cap\textit{Y} = \emptyset$.
\end{enumerate}
Gli insiemi \textit{X} ed \textit{Y} rappresentano i due lati della controversia (i.e. le \textit{echo-chambers}).
\\Per identificare le \textit{echo-chambers}, nell'implementazione proposta è stato utilizzato l'algoritmo di \textit{graph-partitioning} di \textit{Girvan-Newman}. Tale algoritmo agisce rimuovendo progressivamente archi dal grafo originario: l'esecuzione viene arrestata quando la rimozione degli archi ha portato ad individuare due comunità distinte che non comunicano (i.e. non sono collegate da nessun arco). La metrica utilizzata da \textit{Girvan-Newman} per identificare l'arco da rimuovere ad ogni passo è la così detta \textit{edge-betweenness centrality}: dato un arco \textit{e}, essa è definita come \textit{il numero di cammini di costo minimo tra coppie di nodi del grafo che passano attraverso l'arco e}. Nel caso in cui vi sia più di un percorso di costo minimo tra una coppia di nodi, a ciascun percorso viene assegnato uguale peso in modo tale che il peso totale di tutti i percorsi sia uguale all'unità. 
\end{enumerate}

\section{Definizione formale degli algoritmi per la risoluzione}
\section{Calcolo del decremento della controversia associato ad un arco}


\appendix
% INCLUSIONE APPENDICI - - PERSONALIZZARE - TENERE COERENTE CON LISTA IN ALTO
\include{app_a}
\chapter{Another appendix}
\label{app:b}
\Blindtext

%%%%%%%%%%%%%%%%%%%%%%%%%%%%%%%%%%%%%%%%%%%%%%%%%%%%%%%%%%%%%%%

% BIBLIOGRAFIA
\addcontentsline{toc}{chapter}{\refname}
\nocite{*}
\printbibliography

\end{document}
