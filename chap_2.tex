\chapter{Teoria alla base del problema ed algoritmi per la risoluzione}
\label{chap:teoria}
\section{Misura del grado di controversia}
Prima di dare una definizione formale del \textit{random-walk controversy score}, elenchiamo ed illustriamo i passi necessari per calcolarlo.
\begin{enumerate}
\item Fissato il \textit{topic t} per il quale si vuole quantificare il grado di controversia, è possibile descrivere la discussione mediante l'\textit{endorsement graph G(V,E)}. Nell'ambiente di \textit{Twitter}, il \textit{topic t} è identificato da un \textit{hashtag} (e.g.\textit{\#hashtag}) ed i nodi del grafo rappresentano gli utenti che hanno preso parte alla discussione utilizzando almeno una volta tale \textit{hashtag} nei loro \textit{tweets}; gli archi del grafo identificano i \textit{retweets} tra gli utenti, che esprimono relazioni di condivisione di opinione riguardo al \textit{topic}.
\item Ipotizzando che il \textit{topic t} sia controverso, è possibile partizionare i nodi del grafo \textit{G(V,E)} in due insiemi \textit{X},\textit{Y} ben separati tra loro (i.e. vi sono pochi archi che li interconnettono). Tali insiemi quindi soddisfano le seguenti proprietà:
\begin{enumerate}
\item $\textit{X}\cup\textit{Y} = \textit{V}$;
\item $\textit{X}\cap\textit{Y} = \emptyset$.
\end{enumerate}
Gli insiemi \textit{X} ed \textit{Y} rappresentano i due lati della controversia (i.e. le \textit{echo-chambers}).
\\Per identificare le \textit{echo-chambers}, nell'implementazione proposta è stato utilizzato l'algoritmo di \textit{graph-partitioning} di \textit{Girvan-Newman}\cite{girvan:paper}. Tale algoritmo agisce rimuovendo progressivamente archi dal grafo originario: l'esecuzione viene arrestata quando la rimozione degli archi ha portato ad individuare due comunità distinte che non comunicano (i.e. non sono collegate da nessun arco). La metrica utilizzata da \textit{Girvan-Newman} per identificare l'arco da rimuovere ad ogni passo è la così detta \textit{edge-betweenness centrality}: dato un arco \textit{e}, essa è definita come \textit{il numero di cammini di costo minimo tra coppie di nodi del grafo che passano attraverso l'arco e}. Nel caso in cui vi sia più di un percorso di costo minimo tra una coppia di nodi, a ciascun percorso viene assegnato uguale peso in modo tale che il peso totale di tutti i percorsi sia uguale all'unità. Di seguito la formula che definisce questa metrica di centralità:
\\
\begin{equation}b(e) = \sum_{\substack{
   s \neq t
  }} 
 \cfrac{\sigma\textsubscript{st}(e)}{\sigma\textsubscript{st}}
\end{equation}
\\ 
Dove \textit{$\sigma\textsubscript{st}$} è il numero totale di percorsi di costo minimo dal nodo \textit{s} al nodo \textit{t} e \textit{$\sigma\textsubscript{st}(e)$} è il numero di tali percorsi che passano attraverso l'arco \textit{e}. 
\\L'intuizione è: se la struttura del grafo è caratterizzata da due comunità di nodi connesse tra loro da pochissimi archi, allora tutti i percorsi tra queste due comunità dovranno passare attraverso questi archi. Ne consegue che quest'ultimi saranno caratterizzati da un'alta \textit{betweenness centrality}. Sfruttando la peculiarità di tali archi, l'algoritmo di \textit{Girvan-Newman} si rivela un ottimo metodo per rilevare le \textit{echo-chambers}.
\item A questo punto è possibile procedere con la definizione del \textit{random-walk controversy score}. L'\textit{RWC} è definito come \textit{la differenza della probabilità che un random walk che parte da una echo-chamber all'equilibrio vi permanga e la probabilità che invece tale random walk all'equilibrio finisca nell'echo-chamber opposta}. Tale misura viene calcolata mediante l'utilizzo di due esecuzioni dell'algoritmo di \textit{PageRank} personalizzato, le quali non sono altro che due \textit{random walks} particolari.\\
\textit{PageRank} è un algoritmo di analisi che assegna un peso numerico a ciascun nodo di un grafo diretto, con lo scopo di quantificare la sua importanza relativa. Le applicazioni più frequenti di \textit{PageRank} riguardano l'ambito del \textit{World Wide Web}, in cui i grafi hanno come nodi le pagine \textit{web} e come archi i collegamenti ipertestuali. Ciò non toglie che \textit{PageRank} sia uno strumento molto potente anche nell'ambito delle reti sociali, poiché riesce a quantificare l'importanza di utente nell'ambito di una discussione: tale importanza misura il suo grado di popolarità e di rilevanza.
\\Per il calcolo dell'\textit{RWC} vengono utilizzate due esecuzioni distinte dell'algoritmo di \textit{PageRank}, indicate con \textit{page\textsubscript{x} e page\textsubscript{y}}, ognuna delle quali opera sul \textit{retweet graph} corrispondente al \textit{topic t} in input ma inizia il suo \textit{random-walk} partendo, rispettivamente, da uno dei nodi della \textit{comunità X} e da uno dei nodi della \textit{comunità Y} (\textit{comunità = echo-chamber}). Inoltre, \textit{page\textsubscript{x} e page\textsubscript{y}} ad ogni passo possono decidere di continuare il \textit{random-walk} (potendo scegliere con uguale probabilità uno degli archi in uscita dal nodo in cui si trovano attualmente) o di ricominciare il proprio cammino (\textit{restart}), tornando, rispettivamente, in uno dei nodi della \textit{comunità X} ed in uno dei nodi della \textit{comunità Y}: la seconda scelta viene compiuta con una probabilità detta di \textit{restart} e, chiaramente, la prima con una probabilità che ne è il complementare (i.e. la somma delle probabilità deve restituire \textit{1}).
\\Scendendo più nel dettaglio, siano:
\begin{itemize}
\item \textit{P} la \textit{matrice delle probabilità di transizione per colonna}\footnote{Se il \textit{retweet graph} in considerazione ha \textit{N} nodi, tale matrice ha dimensione $N \times N$ ed ogni suo elemento \textit{P[i][j]} è la probabilità di passare dal nodo \textit{j} al nodo \textit{i} in un solo passo, sapendo di essere attualmente nel nodo \textit{j}.} associata al \textit{retweet graph} considerato;
\item \textit{X\textsuperscript{*} e Y\textsuperscript{*}} rispettivamente gli insiemi dei \textit{k\textsubscript{1} e k\textsubscript{2}} nodi con \textit{in-degree} più alto delle due comunità \textit{X e Y}. Inoltre sia \textit{c\textsubscript{x}} un vettore di dimensione \textit{n} avente valore \textit{1} nelle coordinate corrispondenti ai nodi dell'insieme \textit{X\textsuperscript{*}} e \textit{0} altrove; similmente viene definito \textit{c\textsubscript{y}};
\item \textit{r\textsubscript{x}} il vettore di \textit{PageRank} personalizzato per il \textit{random walk} che parte dalla comunità \textit{X}. Sia inoltre $(1-\alpha)$ la probabilità di \textit{restart} di tale \textit{random walk} (e dunque $\alpha$ è la probabilità di continuare) e sia $\textit{e\textsubscript{x} = Uniform(X)}$ il suo vettore di \textit{restart}: ossia il \textit{random walk}, ad ogni passo, decide di ricominciare il proprio cammino con probabilità $(1-\alpha)$ e tra tutti i nodi della \textit{comunità X}, con eguale probabilità, sceglie il nodo da cui ricominciare.
\\Simili considerazioni valgono per \textit{r\textsubscript{y}}.
\end{itemize}
Bisogna ora risolvere il problema dei vertici \textit{dangling}, ossia i vertici del grafo che non hanno archi in uscita. Se un \textit{random walk} dovesse casualmente finire in uno di questi nodi esso potrebbe non uscirne, compromettendo l'esecuzione di \textit{PageRank}. Per evitare tale situazione, vengono utilizzate non una ma due \textit{matrici delle probabilità di transizione per colonna P\textsubscript{x} e P\textsubscript{y}}, usate rispettivamente dal \textit{random walk} che inizia dalla comunità \textit{X} e dal \textit{random walk} che inizia dalla comunità \textit{Y}. Se il grafo non contiene vertici \textit{dangling}, si ha banalmente \textit{P\textsubscript{x} = P\textsubscript{y} = P}; se al contrario li contiene, le matrici \textit{P\textsubscript{x} e P\textsubscript{y}} sono definite in modo tale che le probabilità di transizione dai vertici \textit{dangling} sono uguali, rispettivamente, ai vettori di \textit{restart} \textit{e\textsubscript{x} ed e\textsubscript{y}}.
\\Il \textit{PageRank} personalizzato per i due \textit{random walks} che, rispettivamente, iniziano nella \textit{comunità X e Y} e dato da:
\\
\begin{equation}
r_x = \alpha P_xr_x + (1-\alpha)e_x
\end{equation}
\begin{equation}
r_y = \alpha P_yr_y + (1-\alpha)e_y
\end{equation}
\\
Per il calcolo dei vettori di \textit{PageRank} personalizzati (\textit{r\textsubscript{x} e r\textsubscript{y}}), quindi, bisogna imporre una situazione di \textit{stazionarietà} dei rispettivi \textit{random walks}. 
\\Possiamo finalmente definire il \textit{random-walk controversy score}:
\\
\begin{equation}
RWC(G,X,Y) = (c_x - c_y)\textsuperscript{T}(r_x - r_y)
\end{equation}
\\
Sostituendo le equazioni (2.2) e (2.3) nell'espressione (2.4) si ottiene:
\\
\begin{equation}
RWC(G,X,Y) = (1 - \alpha)(c_x - c_y)\textsuperscript{T}(M_x\textsuperscript{-1}e_x - M_y\textsuperscript{-1}e_y)
\end{equation}
\\
Dove \textit{M\textsubscript{x} = (I - $\alpha$P\textsubscript{x})} e \textit{M\textsubscript{y} = (I - $\alpha$P\textsubscript{y})}. Quest'ultima è la formula utilizzata nell'implementazione del sistema proposto. Ad ogni modo, riferendosi all'equazione (2.4) è possibile fare le seguenti osservazioni:
\begin{enumerate}
\item Innanzitutto occorre precisare che \textit{r\textsubscript{x}} è una \textit{distribuzione stazionaria di probabilità} del \textit{random walk} associato all'esecuzione di \textit{PageRank page\textsubscript{x}}: ognuno dei suoi elementi \textit{r\textsubscript{x}[i]} ($\forall \textit{i = 0,...,n-1}$) rappresenta la probabilità che il \textit{random walk} si trovi nel nodo \textit{i} all'equilibrio. Simili considerazioni valgono per \textit{r\textsubscript{y}};
\item Qualora le due comunità (\textit{echo-chambers}) \textit{X e Y} fossero molto divise tra loro (i.e. pochissimi archi diretti che le connettono), ci si aspetterebbe un \textit{r\textsubscript{x}} con valori pressoché nulli nelle coordinate corrispondenti ai nodi di \textit{Y} e non nulli altrove e, viceversa, parlando di \textit{r\textsubscript{y}}\footnote{Ricordare che, essendo \textit{distribuzioni di probabilità}, vale: $\sum_{i=0}^{n-1} r_x[i] = 1$ e $\sum_{j=0}^{n-1} r_y[j] = 1$.}. In tal caso l'espressione $(c_x - c_y)\textsuperscript{T}(r_x - r_y)$ assume il valore massimo, indice di un'elevata \textit{controversia} nella rete.
\item Qualora le due comunità fossero invece sufficientemente connesse tra loro, ci si aspetterebbe un vettore \textit{r\textsubscript{x}} con valori abbastanza uniformi su tutte le sue coordinate; la stessa considerazione varrebbe per \textit{r\textsubscript{y}}. In tal caso il grafo non presenterebbe un'elevata controversia (\textit{RWC(G,X,Y)} assume valori bassi), e questo risultato sarebbe confermato dal fatto che le due comunità sono sufficientemente esposte l'una a l'altra.
\end{enumerate}
\end{enumerate}
Per terminare, si può affermare che l'indice \textit{RWC(G,X,Y)} riesce a descrivere perfettamente il grado di controversia della discussione in atto, in funzione del livello di esposizione reciproca delle due comunità (ossia i due punti di vista opposti): qualora tale livello fosse molto basso, l'indice di controversia sarebbe molto elevato e, pertanto, la scelta più efficace per attenuarlo sarebbe quella di cercare di esporre gli utenti a visioni opposte alle proprie. 
\\Nel prossimo paragrafo saranno illustrati gli algoritmi utilizzati nel sistema proposto, i quali hanno come obiettivo proprio quello di esporre reciprocamente, \textit{nel modo più efficace possibile}, i due lati della controversia. 

\section{Definizione formale degli algoritmi per la risoluzione}
Il problema che si vorrebbe risolvere è il seguente: 
\\\\
\textit{Dato un endorsement graph G(V,E) che descrive una discussione riguardo ad un certo topic controverso t, trovare l'insieme di k archi diretti non ancora presenti nel grafo che, qualora si materializzassero, minimizzerebbero il suo RWC(G,X,Y).}
\\\\
Ovvero, formalmente:
\\
\begin{equation*}
\begin{aligned}
& \underset{E\textsubscript{k}}{\text{minimize}}
& & RWC(G(V, E \cup E\textsubscript{k}),X,Y) \\
& \text{subject to}
& & E\textsubscript{k} \subseteq V \times V \setminus E, \left|{E\textsubscript{k}}\right| = k
\end{aligned}
\end{equation*}
\\
Risolvere tale problema, così come si presenta, richiederebbe di considerare tutti le possibili combinazioni degli archi ancora non presenti nel grafo presi a gruppi di \textit{k}: queste combinazioni sono pari a \textit{O}($n^2 \choose k$), dove con \textit{n} indichiamo il numero di nodi del grafo.
\\Come è facile immaginare, un algoritmo che considera tutte queste combinazioni di archi è molto inefficiente dal punto di vista computazionale, soprattutto alla luce del fatto che gli \textit{endorsement graphs} delle reti sociali come quella di \textit{Twitter} sono costituiti da alcune migliaia di nodi e migliaia di archi.
\\Si propone, pertanto, di restringere il dominio degli archi candidati: seguendo l'approccio dell'articolo \cite{garimella:paper} è possibile considerare solo gli archi tra vertici con \textit{in-degree} alto di ciascuna \textit{echo-chamber}. La scelta di questa euristica, come osservano gli autori dell'articolo \cite{garimella:paper}, è giustificata dal fatto che la struttura degli \textit{endorsement graph} spesso mette in luce la presenza di un piccolo numero di nodi \textit{leader} (i.e. utenti popolari) ed un gran numero di nodi \textit{non leader} (i.e. utenti seguaci): i nodi \textit{leader} ricevono molti archi in ingresso (ossia hanno alto \textit{in-degree}) in quanto i loro \textit{tweet} vengono \textit{retweettati} da molti nodi \textit{non leader} (ossia ricevono molta \textit{approvazione} dai loro seguaci, i quali si fidano della loro opinione). Intuitivamente, qualora si riuscisse a convincere i nodi \textit{leader} ad approvare contenuti che esprimono opinioni sul \textit{topic controverso} opposte alla propria, i nodi \textit{non leader} (i seguaci) sarebbero spinti a fare altrettanto.
Questa osservazione suggerisce che gli archi tra vertici con \textit{in-degree} alto di ciascuna \textit{echo-chamber} sono buoni candidati per ottenere un abbassamento del \textit{random-walk controversy score}\footnote{Si rimanda alla lettura del \textit{Teorema 1.} dell'articolo \cite{garimella:paper}.}. 
\\Mediante questo ridimensionamento del dominio degli archi candidati, ci si propone di ridurre l'\textit{RWC} esponendo alcuni nodi \textit{leader} di ciascuna \textit{echo-chamber} ai contenuti di alcuni nodi \textit{leader} dell'\textit{echo-chamber} opposta. 
\\Ora, con l'obiettivo di effettuare la scelta dei \textit{k} archi migliori appartenenti al dominio così definito, vengono proposti due algoritmi: un \textit{algoritmo non greedy} (indicato di seguito come \textit{Algorithm 1}) ed un \textit{algoritmo greedy}.
\\L'algoritmo \textit{non greedy} ha un tempo di esecuzione pari a \textit{O($k_1 \cdot k_2$)}, che costituisce un ottimo \textit{speedup} rispetto all'algoritmo \textit{brute-force}, che considera invece tutte le combinazioni di archi. \textit{Algorithm 1} si limita a scegliere dal dominio degli archi, ridimensionato come appena detto, i \textit{k} archi migliori in termini del decremento dell'\textit{RWC} che ciascuno consente se aggiunto \textit{individualmente}. Tuttavia il \textit{decremento dell'RWC} che un qualsiasi arco \textit{"e"} consente individualmente (\textit{$\delta$RWC\textsubscript{e}}) si rivelerebbe tale solo se tale arco fosse aggiunto per primo al grafo; più precisamente $\forall E' \supseteq E, \forall e \in V \times V \setminus E'$ vale l'espressione (2.6). 
\begin{equation}
\resizebox{1 \textwidth}{!} 
{
    $ RWC(G(V,E),X,Y) - RWC(G(V, E \cup \{e\}),X,Y) \geq RWC(G(V,E'),X,Y) - RWC(G(V, E' \cup \{e\}),X,Y)$
} 
\end{equation}
Ovvero la somma dei \textit{$\delta$RWC} dei \textit{k} archi scelti dall'algoritmo \textit{non greedy} non corrisponde al decremento \textit{reale} dell'\textit{RWC} che si osserverebbe se tali archi apparissero nel grafo, ma ne è un \textit{upper-bound}. Questo significa che l'algoritmo \textit{non greedy} è poco preciso nella scelta e potrebbe pertanto disattendere le aspettative di decremento della \textit{controversia}.
\\L'alternativa è l'utilizzo di una versione \textit{greedy} di tale algoritmo. L'algoritmo \textit{greedy} non fa altro che effettuare la scelta dei \textit{k} archi non in un solo \textit{step} ma in \textit{k step}. In ognuno dei \textit{k} passi sceglie uno ed un solo arco, ossia l'arco migliore, tra quelli ancora disponibili, in termini del $\delta$RWC che consentirebbe se fosse aggiunto al grafo: l'arco scelto, infine, viene aggiunto al grafo con l'obiettivo di consentire una scelta più precisa dei restanti archi.
\\Si osserva che, per la versione \textit{greedy}, è vero che la somma dei \textit{$\delta$RWC} dei \textit{k} archi scelti corrisponde al decremento \textit{reale} dell'\textit{RWC} che si osserverebbe se tali archi apparissero nel grafo. Nel capitolo relativo ai \textit{test} sarà possibile osservare quanto la versione \textit{greedy} riesca ad individuare archi migliori, in termini del decremento dell'\textit{RWC}, rispetto a quelli individuati dalla versione \textit{non greedy}.
\\Lo svantaggio della versione \textit{greedy} riguarda il tempo di esecuzione. Il fatto che tale algoritmo sia sintetizzabile come un algoritmo \textit{non greedy} eseguito \textit{k} volte implica che il suo tempo di esecuzione sia pari a \textit{O($k \cdot k_1 \cdot k_2$)}, ovvero \textit{k} volte il tempo di esecuzione della versione \textit{non greedy}. Ad ogni modo, la scelta dell'algoritmo di \textit{edge recommendation} da utilizzare dovrebbe sempre essere dettata da un giusto compromesso tra i risultati che consente di ottenere e l'efficienza nei tempi di esecuzione.
\\\\
Nel prossimo paragrafo sarà illustrata una tecnica efficiente per calcolare il \textit{$\delta$RWC} associato a ciascun arco considerato dagli algoritmi appena descritti.
\begin{algorithm}
\caption{Algoritmo \textit{non greedy} per la scelta dei \textit{k} archi}
\begin{algorithmic} 
\REQUIRE Il grafo \textit{G} e le comunità \textit{X,Y}; il numero di archi da proporre \textit{k}; i \textit{k\textsubscript{1} e k\textsubscript{2}} vertici con \textit{in-degree} più alto in \textit{X e Y}, rispettivamente 
\ENSURE La lista dei \textit{k} archi migliori, in termini del decremento dell'\textit{RWC} che consentono se aggiunti individualmente
\STATE Initialization : Output $\leftarrow$ lista vuota;
\FOR{\textit{i = 1:k\textsubscript{1}}} 
\STATE nodo $u = X[i]$;
\FOR{\textit{j = 1:k\textsubscript{2}}}
\STATE nodo $v = Y[j]$;
\STATE Calcola il decremento dell'\textit{RWC $\delta$RWC\textsubscript{u $\rightarrow$ v}} che si otterrebbe qualora l'arco \textit{(u,v)} venisse aggiunto al grafo;
\STATE Aggiungi \textit{$\delta$RWC\textsubscript{u $\rightarrow$ v}} alla lista \textit{Output};
\STATE Calcola il decremento dell'\textit{RWC $\delta$RWC\textsubscript{v $\rightarrow$ u}} che si otterrebbe qualora l'arco \textit{(v,u)} venisse aggiunto al grafo;
\STATE Aggiungi \textit{$\delta$RWC\textsubscript{v $\rightarrow$ u}} alla lista \textit{Output};
\ENDFOR
\ENDFOR
\STATE Lista \textit{Output} ordinata $\leftarrow$ Ordina la lista \textit{Output} secondo i \textit{$\delta$RWC} (cambiati di segno) in ordine decrescente;
\RETURN I migliori \textit{k} dalla lista \textit{Output} ordinata;
\end{algorithmic}
\end{algorithm}

\section{Calcolo del decremento della controversia associato ad un arco}

Per ogni arco \textit{$e \in V \times V \setminus E$} appartenente al dominio dei \textit{$k_1 \times k_2$} archi considerati dagli algoritmi \textit{non greedy e greedy}, si rende necessario calcolare il \textit{$\delta$RWC\textsubscript{e}} che esso consentirebbe qualora venisse aggiunto al grafo. Per effettuare questo calcolo verrebbe in mente di aggiungere l'arco \textit{"e"} al grafo, calcolare il nuovo \textit{RWC\textsubscript{\{e\}}}, per poi ricavare: 

\begin{equation}
\textit{$\delta$RWC\textsubscript{e}} = \textit{RWC - RWC\textsubscript{\{e\}}}
\end{equation}

\`E bene ricordare che il calcolo dell'\textit{RWC} è abbastanza oneroso e pertanto è meglio evitarlo quando possibile: se si utilizzasse la modalità (2.7), gli algoritmi di scelta dei \textit{k} archi dovrebbero calcolare l'\textit{RWC} un numero di volte pari a \textit{$k_1 \times k_2$}, cagionando un degrado prestazionale non trascurabile.
\\Tuttavia, visto che siamo interessati solamente all'entità del decremento dell'\textit{RWC} a seguito dell'aggiunta di un arco, possiamo utilizzare un'altra modalità di calcolo più efficiente. In particolare, considerando la \textit{matrice delle probabilità di transizione per colonna P}, dopo l'aggiunta del generico arco diretto \textit{(i,j)} solo una sua colonna ne è affetta: la colonna che corrisponde al vertice di origine \textit{"i"} dell'arco \textit{(i,j)}.
\\Di seguito è evidenziata la colonna \textit{i-esima} prima e dopo l'aggiunta dell'arco diretto \textit{(i,j)}:

$P^T = \begin{bmatrix}
       \cdots & \cdots & \cdots & \cdots & \cdots & \cdots & \cdots & \cdots \\[0.3em]
       \cdots & \cdots & \cdots & \cdots & \cdots & \cdots & \cdots & \cdots \\[0.3em]
       \frac{1}{q} & \frac{1}{q} & \cdots & \frac{1}{q} & 0 & 0 & \cdots & 0 \\[0.3em]
       \cdots & \cdots & \cdots & \cdots & \cdots & \cdots & \cdots & \cdots \\[0.3em]
       \cdots & \cdots & \cdots & \cdots & \cdots & \cdots & \cdots & \cdots \\[0.3em]
     \end{bmatrix}$
\\\\
Dove \textit{q} è l'\textit{out-degree} del nodo \textit{i}. A seguito dell'aggiunta dell'arco diretto \textit{(i,j)} la colonna \textit{i-esima} diventa:
\\\\
$P'^T = \begin{bmatrix}
       \cdots & \cdots & \cdots & \cdots & \cdots & \cdots & \cdots & \cdots \\[0.3em]
       \cdots & \cdots & \cdots & \cdots & \cdots & \cdots & \cdots & \cdots \\[0.3em]
       \frac{1}{q+1} & \frac{1}{q+1} & \cdots & \frac{1}{q+1} & \frac{1}{q+1} & 0 & \cdots & 0 \\[0.3em]
       \cdots & \cdots & \cdots & \cdots & \cdots & \cdots & \cdots & \cdots \\[0.3em]
       \cdots & \cdots & \cdots & \cdots & \cdots & \cdots & \cdots & \cdots \\[0.3em]
     \end{bmatrix}$
\\\\
L'elemento in posizione \textit{j-esima} della colonna \textit{i-esima} della nuova \textit{matrice di transizione P'} passa da un valore pari a \textit{0} ad un valore pari a $\frac{1}{q+1}$.
\\Sia ora \textit{u\textsuperscript{T}} l'\textit{i-esimo} vettore della base standard di ${\rm I\!R^n}$ (dove \textit{n} è il numero di nodi del grafo); similmente, sia \textit{v\textsuperscript{T}} il \textit{j-esimo} vettore della base standard di ${\rm I\!R^n}$. 
\\Definiamo infine il vettore \textit{z\textsuperscript{T}} come segue:
\begin{enumerate}
\item Se il vertice \textit{i} non è un vertice \textit{dangling}, $\textit{z\textsuperscript{T}} = \frac{1}{q+1}[\frac{1}{q},\frac{1}{q},...,\frac{1}{q},-1,...0,0,0]$, con \textit{-1} nella posizione corrispondente al vertice di arrivo \textit{j};
\item Se il vertice \textit{i} è un vertice \textit{dangling}, $\textit{z\textsuperscript{T}} = e_x - v$ oppure $\textit{z\textsuperscript{T}} = e_y - v$, rispettivamente se si considera la matrice $P_x$ (\textit{random walk} che parte dall'\textit{echo-chamber X}) o la matrice $P_y$ (\textit{random walk} che parte dall'\textit{echo-chamber Y}).
\end{enumerate}
Si dimostra che la \textit{matrice delle probabilità di transizione} aggiornata è data da:

\begin{equation}
P' = P - z \otimes u^T
\end{equation}
\\
Dove il simbolo $\otimes$ indica che il prodotto tra i due vettori non è scalare ma \textit{esterno}, il cui risultato è una matrice.
\\Ricordando dalla definizione dell'\textit{RWC} (2.5) le formule che descrivono le matrici $M_x$ ed $M_y$ e sostituendovi, rispettivamente, $P'_x$ e $P'_y$ (seguendo la formula (2.8)) si ottiene:
\\
\begin{equation}
M'_x = M_x + \alpha z_x \otimes u^T
\end{equation}
\begin{equation}
M'_y = M_y + \alpha z_y \otimes u^T
\end{equation}
\\
Nell'implementazione proposta, l'inversa delle matrici $M'_x$ e $M'_y$ (inversa che occorre per il calcolo dell'\textit{RWC}) è calcolata usando la formula di \textit{Sherman-Morrison}\cite{golub:paper}, vista la sua efficienza.
\\Ora, per l'equazione (2.5), l'\textit{RWC} del grafo a seguito dell'aggiunta del nuovo arco diretto può essere scritto come:
\\
\begin{equation}
RWC' = (1 - \alpha)(c_x - c_y)\textsuperscript{T}(M'_x\textsuperscript{-1}e_x - M'_y\textsuperscript{-1}e_y)
\end{equation}
\\
E, finalmente, il $\delta$RWC è dato da:
\\
\begin{equation}
\begin{split}
\delta RWC = RWC' - RWC = (1 - \alpha)(c_x - c_y)\textsuperscript{T}(-(\frac{\alpha M_x\textsuperscript{-1}z_x \otimes u^T M_x\textsuperscript{-1}}{1 + \alpha u^T M_x\textsuperscript{-1}z_x})e_x \\+ (\frac{\alpha M_y\textsuperscript{-1}z_y \otimes u^T M_y\textsuperscript{-1}}{1 + \alpha u^T M_y\textsuperscript{-1}z_y})e_y)
\end{split}
\end{equation}
\\ 
Quest'equazione, sebbene apparentemente ingombrante, permette di calcolare in modo molto efficiente il $\delta RWC$ di ogni arco considerato, evitando di dover calcolare ogni volta il nuovo \textit{RWC} (operazione molto costosa) per poi prendere la differenza rispetto all'\textit{RWC} precedente. 
\\\\
Nel prossimo capitolo saranno presentate le tecnologie utilizzate per realizzare il sistema proposto e verranno illustrati i dettagli implementativi.

